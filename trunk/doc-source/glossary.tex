%% -*- Mode:TeX; Fonts:(hl12fb) -*-
%% *-* File: /usr/local/gbbopen/doc-source/glossary.tex *-*
%% *-* Last-Edit: Sat Jun 20 10:29:46 2009; Edited-By: cork *-*
%% *-* Machine: cyclone.cs.umass.edu *-*

\begin{glossary-list}
  
%% ------------------------------------------------------------------------
  
\glent[alist]
\index{alist|see{association list}}%
%
An \glref{association~list}.

%% ------------------------------------------------------------------------

\glent[association~list]
\index{association list}%
%
A list of \glref{conses} representing an association of keys with values. The
\glref{car} of each \glref{cons} is the key and the \glref{cdr} is the value
associated with that key.

%% ------------------------------------------------------------------------

\glent[atomic~operation]
\index{atomic operation}%
%
A computation that, once started, is completed without being interrupted by
another \glref{thread}.

%% ------------------------------------------------------------------------

\glent[autorun~forms]
\index{autorun form, module}%
%
Forms in a \glref{module} file that are to be evaluated conditionally when the
file is loaded based on the value of \nobr{\textbf{\entlink{*autorun-modules*}}}.
Part of the \reflink{Module Manager Facility}{sec:module-manager}.

%% ------------------------------------------------------------------------

\glent[blackboard~repository]
\index{blackboard repository}%
%
The internal storage containing all \glref{unit~instance} and
\glref{space~instance} objects and associated retrieval data structures.

%% ------------------------------------------------------------------------

\glent[boolean~dimension]
\gllabel{boolean~dimensions}%
\index{boolean dimension}%
\index{dimension type!boolean}%
%
A \glref{dimension} of \code{:boolean} \glref{dimension~type} where
\code{:boolean} \glref{dimension~values} are either true (non-\nil) or
false (\nil).

%% ------------------------------------------------------------------------

\glent[circular~list]
\index{circular list}%
%
A list that has no termination because it includes an earlier portion of
itself in its successive sublists.

%% ------------------------------------------------------------------------

\glent[class]
\gllabel{classes}%
\index{class}%
%
An object that uniquely (directly or indirectly) determines the structure and
behavior of a set of other objects. Members of this set are called
\glref{instances} of the class.

%% ------------------------------------------------------------------------

\glent[class~designator]
\index{class designator}%
\indexit{class}%
\indexit{class object}%
%
A \glref{class} or a symbol that names a \glref{class}.

%% ------------------------------------------------------------------------

\glent[class~option]
\index{class!option}%
%
An option that refers to a \glref{class} as a whole or to all the slots of the
\glref{class}.

%% ------------------------------------------------------------------------

\glent[comparison~type]
\gllabel{comparison-type}
\index{comparison-type, of dimension values}%
\index{dimension value!comparison-type}%
\codeindexit{number}%
\codeindexit{fixnum}%
\codeindexit{short-float}%
\codeindexit{single-float}%
\codeindexit{double-float}%
\codeindexit{double-float}%
\codeindexit{long-float}%
\codeindexit{eq}%
\codeindexit{eql}%
\codeindexit{equal}%
\codeindexit{equalp}%
%
A symbol that indicates the way that \glref{dimension~values} are compared.
For an \code{:ordered} dimension value, one of: \code{number},
\code{fixnum}, \code{short-float}, \code{single-float}, \code{double-float},
or \code{long-float}.  For an \code{:enumerated} dimension value, one of:
\code{eq}, \code{eql}, \code{equal}, or \code{equalp}.  For a \code{:boolean}
dimension value: \code{t}.

%% ------------------------------------------------------------------------

\glent[composite~dimension~value]
\index{composite dimension!value}% 
%
A dimension value that is a set, sequence, or series of
\glref{dimension~values}.

%% ------------------------------------------------------------------------

\glent[condition~variable]
\gllabel{condition~variables}%
\gllabel{condition-variable}%
\index{POSIX-style, condition variable}% 
%
A condition variable provides an atomic means for a \glref{thread} to release
a \glref{lock} (or \glref{recursive~lock}) that it holds and go to sleep until
it is awakened by another thread.  Once awakened, the lock that it was holding
is reacquired atomically before the awakened thread is allowed to do anything
else.  A Portable Threads condition-variable object includes the lock that is
associated with the condition variable, and the condition-variable object can
be used directly as a lock.

%% ------------------------------------------------------------------------

\glent[cons]
\gllabel{car}%
\gllabel{cdr}%
\gllabel{conses}%
\bfindex{cons}%
\bfindex{car}%
\bfindex{cdr}%
%
An object with two components called the \textbf{car} and the \textbf{cdr}.
Conses are used to construct lists.

%% ------------------------------------------------------------------------

\glent[dimension] 
\gllabel{dimensions}%
%
A conceptual extent within which values that share some
relationship can be placed.  GBBopen uses dimensionality to relate the
extent representations of \glref{unit~instances},
\glref{space~instances}, and \glref{retrieval~patterns}.

GBBopen supports three \glref{dimension~types}:
\begin{tightenumerate}
\item \glref{ordered~dimensions} (\code{:ordered})
\item \glref{enumerated~dimensions} (\code{:enumerated})
\item \glref{boolean~dimensions} (\code{:boolean})
\end{tightenumerate}

Real-world dimensions (such as time and location) can be represented as
\glref{ordered~dimensions}.

%% ------------------------------------------------------------------------

\glent[dimension~name]
\gllabel{dimension-name}%
\index{dimension name}%
%
A symbol used to identify a \glref{dimension}.  In general, two
\glref{dimensions} with different \glref{dimension~types} should not be given
the same dimension name.

%% ------------------------------------------------------------------------

\glent[dimension~type]
\gllabel{dimension-type}%
\gllabel{dimension~types}%
\index{dimension type}%
\index{type!dimension}%
%
The interpretation associated with a \glref{dimension} of a
\glref{unit~class}, a \glref{space~instance}, or a \glref{retrieval~pattern};
one of \code{:ordered}, \code{:enumerated}, or \code{:boolean}.

%% ------------------------------------------------------------------------

\glent[dimension~value]
\gllabel{dimension~values}%
\index{dimension value}%
%
The value that can be used to position a \glref{unit~instance} on a
\glref{dimension} in one or more \glref{space~instances}.

%% ------------------------------------------------------------------------

\glent[dimension-value~type]
\gllabel{dimension-value~types}%
\index{dimension value!type}%
\index{type!dimension value}%
\codeindexit{:point}%
\codeindexit{:interval}%
\codeindexit{:mixed}%
\codeindexit{:element}%
\codeindexit{:boolean}%
%
The interpretation associated with a \glref{dimension~value} of a
\glref{unit~instance}; one of \code{:point}, \code{:interval}, \code{:mixed}
(both points and intervals), \code{:element}, or \code{:boolean}.  
The dimension-value types \code{:point}, \code{:interval}, and
\code{:mixed} indicate values in an \glref{ordered~dimension},
\code{:element} indicates a value in an \glref{enumerated~dimension}, and
\code{:boolean} indicates a value in a \glref{boolean~dimension}.

%% ------------------------------------------------------------------------

\glent[dimensional~extent]
\index{dimensional extent, of a space instance}%
\index{space instance!dimensional extent}%
%
The \glref{dimensions} of a \glref{space~instance}.

%% ------------------------------------------------------------------------

\glent[dotted~list]
\index{dotted list}%
\index{list!dotted}%
%
A list that is terminated by a non-\nil{} atom rather than the empty list,
\nil.

%% ------------------------------------------------------------------------

\glent[enumerated~dimension]
\gllabel{enumerated~dimensions}%
\index{enumerated dimension}%
\index{dimension type!enumerated}%
%
A \glref{dimension} of \code{:enumerated} \glref{dimension~type} where
\code{:element} \glref{dimension~values} are individual elements from an
extensible set of discrete elements.

%% ------------------------------------------------------------------------

\glent[event]
\gllabel{events}%
\index{event}%
%
An activity that is noticed, and signaled, by GBBopen.

%% ------------------------------------------------------------------------

\glent[event~class]
\gllabel{event-class}%
\gllabel{event~classes}%
\index{event class}%
\index{class!event}%
\bfindexit{standard-event-class}%
%
An object that is a subclass of \nobr{\textbf{\entlinknoex{standard-event-class}}}.

%% ------------------------------------------------------------------------

\glent[event~function]
\gllabel{event-function}%
\gllabel{event~functions}%
\index{event function}%
\index{function!event}%
%
A \glref{function} that is associated with one or more event specifications
and is called whenever such an event occurs.  (See
\nobr{\textbf{\entlink{signal-event}}} for the required event-function arguments for
each event metaclass.

%% ------------------------------------------------------------------------

\glent[event~instance] 
\gllabel{event-instance}%
\gllabel{event~instances}%
\index{event instance}%
\index{instance!event}%
\bfindexit{standard-event-instance}%
%
An object whose class is a subclass of
\nobr{\textbf{\entlinknoex{standard-event-instance}}}.

%% ------------------------------------------------------------------------

\glent[event~metaclass]
\gllabel{event-metaclass}%
\index{event metaclass}%
\index{metaclass!event}%
%
One of five ``types'' of \glref{event}.  Every \glref{event~class} has
one of the following event metaclasses:
\code{non-instance-event-class}, \code{instance-event-class},
\code{space-instance-event-class}, \code{nonlink-slot-event-class}, or
\code{link-slot-event-class}.

%% ------------------------------------------------------------------------

\glent[executable~KS~activation]
\gllabel{executable~KSA}%
\gllabel{executable~KSAs}%
\index{executable knowledge-source activation}%
\index{KS activation!executable}%
\index{KSA!executable}%
%
A \glref{pending~KS~activation} that meets the criteria for execution, such as
having a \glref{rating} above the minimum KSA execution rating in effect for
the control shell.

%% ------------------------------------------------------------------------

\glent[executed~KS~activation]
\gllabel{executed~KSA}%
\gllabel{executed~KSAs}%
\index{executed knowledge-source activation}%
\index{KS activation!executed}%
\index{KSA!executed}%
%
A \glref{KS~activation} that has completed execution and will, therefore, not
be operated on again by the control shell.

%% ------------------------------------------------------------------------

\glent[extended~event-class~specification]
\index{extended event-class specification}%
\index{event class!extended event-class specification}%
%
A specification of one or more \glref{event~classes} as indicated by
one of the following:
\begin{tightitemize}
\item a \glref{event~class}
\item a symbol naming a \glref{event~class}
\item
\codeindex{:plus-subevents}%
\codeindex{:no-subevents}%
a list containing one of the above followed by the \glref{keyword}
\code{:plus-subevents} or the \glref{keyword} \code{:no-subevents}
\item the symbol \code{t}, which is equivalent to
  \code{(standard-event-instance :plus-subevents)} 
\end{tightitemize}

%% ------------------------------------------------------------------------

\glent[extended~unit-class~specification]
\index{extended unit-class specification}%
\index{unit class!extended unit-class specification}%
%
A specification of one or more \glref{unit~classes} as indicated by
one of the following:
\begin{tightitemize}
\item a \glref{unit~class}
\item a symbol naming a \glref{unit~class}
\item
\codeindex{:plus-subclasses}%
\codeindex{:no-subclasses}%
a list containing one of the above followed by the \glref{keyword}
\code{:plus-subclasses} or the \glref{keyword} \code{:no-subclasses}
\item the symbol \code{t}, which is equivalent to
  \code{(standard-unit-instance :plus-subclasses)} 
\end{tightitemize}

%% ------------------------------------------------------------------------

\glent[extended~unit-class~or~instance~specification]
\index{extended unit-class specification}%
\index{unit instance!specification}%
\index{unit class!extended unit-class specification}%
%
A specification of one or more \glref{unit~instances} or one or more
\glref{unit~classes} as indicated by one of the following:
\begin{tightitemize}
\item a \glref{unit~class}
\item a symbol naming a \glref{unit~class}
\item
\codeindex{:plus-subclasses}%
\codeindex{:no-subclasses}%
a list containing one of the above followed by the \glref{keyword}
\code{:plus-subclasses} or the \glref{keyword} \code{:no-subclasses}
\item the symbol \code{t}, which is equivalent to
  \code{(standard-unit-instance :plus-subclasses)} 
\item a \glref{unit~instance}
\item a list of \glref{unit~instances}
\end{tightitemize}

%% ------------------------------------------------------------------------

\glent[extended~unit-classes~specification]
\gllabel{extended~unit-classes~specifications}%
\index{extended unit-classes specification}%
\index{unit class!extended unit-classes specification}%
%
A specification of one or more \glref{unit~classes} as indicated by
one of the following:
\begin{tightitemize}
\item a \glref{unit~class}
\item a symbol naming a \glref{unit~class}
\item
\codeindex{:plus-subclasses}%
\codeindex{:no-subclasses}%
a list containing one of the above followed by the \glref{keyword}
\code{:plus-subclasses} or the \glref{keyword} \code{:no-subclasses}
\item a list of one or more of the above
\item the symbol \code{t}, which is equivalent to
  \code{(standard-unit-instance :plus-subclasses)} 
\end{tightitemize}

%% ------------------------------------------------------------------------

\glent[feature]
\index{feature}%
\bfindexit{*features*}%
%
A symbol in the list value of the variable \textbf{*features*}.  The features
in this features list are used to control conditional compilation and
implementation-specific behaviors.

%% ------------------------------------------------------------------------

\glent[form]
\gllabel{forms}%
\gllabel{Form}%
\gllabel{forms}%
\index{form}%
%
An object (including an expression) to be evaluated.

%% ------------------------------------------------------------------------

\glent[function~designator]
\gllabel{function}%
\index{function designator}%
\indexit{function}%
\indexit{function object}%
%
An object that specifies a function.  Either: a symbol (denoting the function
named by that symbol in the global environment), or a \glref{function~object}
(denoting itself).  The term ``function'' is often used to denote a
function designator, with the term ``function object'' used when referring
specifically to a \glref{function~object}.

%% ------------------------------------------------------------------------

\glent[function~object]
\index{function object}%
%
An object of type \code{function}.  The term ``function'' is often
used to denote a \glref{function~designator}, with the term ``function
object'' used when referring specifically to a function object.

%% ------------------------------------------------------------------------

\glent[fixnum]
\gllabel{fixnums}%
\index{fixnum}%
%
An integer between \code{most-negative-fixnum} and
\code{most-positive-fixnum} inclusive.

%% ------------------------------------------------------------------------

\glent[generalized~boolean]
\gllabel{generalized-boolean}%
\index{generalized boolean}%
\index{boolean, generalized}%
%
An object used as a truth value, where \nil{} represents false and all other
objects represent true.

%% ------------------------------------------------------------------------

\glent[generalized~reference]
\index{generalized reference}%
\index{reference, generalized}%
%
A reference to a location storing a value as if to a variable.

%% ------------------------------------------------------------------------

\glent[generic~function]
\index{generic function}%
%
A \glref{function} whose behavior depends on the classes or identities of the
arguments supplied to it.

%% ------------------------------------------------------------------------

\glent[incomposite~dimension~value]
\index{incomposite dimension value}%
\index{dimension value!incomposite}%
%
A \glref{dimension~value} that is a single point, interval, element, or
boolean (i.e., not a \glref{composite~dimension~value}).

%% ------------------------------------------------------------------------

\glent[initialization~argument~list]
\gllabel{initialization~arguments}%
\index{initialization argument list}%
\index{list!initialization arguments}%
%
A list of alternating names and values used to initialize or reinitialize
\glref{instances} of \glref{classes}.  If more than one name and value pair
has the same name, only the first such pair is used to provide the value.

%% ------------------------------------------------------------------------

\glent[instance]
\gllabel{instances}%
\index{instance}%
%
An object whose structure and behavior is uniquely (directly or indirectly)
determined by a \glref{class} object.

%% ------------------------------------------------------------------------

\glent[instance~name]
\gllabel{instance-name}%
\gllabel{instance~names}%
\index{instance name}%
%
An object that uniquely identifies an \glref{instance} of a
\glref{unit~class}.  The same object can be used to identify instances of
different unit classes, but the same instance name cannot be used with two
instances of the same class.

%% ------------------------------------------------------------------------

\glent[interval] 
\gllabel{intervals}%
\index{interval}%
%
A \glref{cons}, two-element list, or two-element array containing the start
and end value representing the set of real numbers between them, inclusive.

%% ------------------------------------------------------------------------

\glent[keyword~symbol]
\gllabel{keyword}%
\gllabel{keywords}%
\index{keyword symbol}%
\index{symbol!keyword}%
%
A symbol whose home package is the \code{keyword} package.

%% ------------------------------------------------------------------------

\glent[knowledge~source]
\gllabel{knowledge-source}%
\gllabel{KS}%
\gllabel{KSs}%
\index{knowledge source}%
%
The expertise associated with a collaborating computational entity in
a blackboard application (often abbreviated as ``KS''). More
specifically, a KS is an object containing the expertise and other
information associated with a computational entity. A \code{ks} object
is also a \glref{unit~instance}, but KSs are normally described
by their more specific categorization.

%% ------------------------------------------------------------------------

\glent[KS~activation]
\gllabel{KSA}%
\gllabel{KSAs}%
\gllabel{KS~activation}%
\gllabel{KS~activations}%
\gllabel{activation}%
\gllabel{activations}%
\index{knowledge source!activation}%
\index{KS!activation}%
%
The application of a \glref{KS} to a specific computational context
(often abbreviated as ``KSA'').  More specifically, a KSA is a
\glref{ksa-class} object representing the \glref{KS} activation.  A
\code{ksa} is also a \glref{unit~instance}, but they are normally
described by their more specific categorization.

%% ------------------------------------------------------------------------

\glent[KS~execution]
\gllabel{KS-execution}%
\gllabel{KS~executions}%
\index{knowledge source!execution}%
\index{KS!execution}%
%
The execution of a \glref{KS~activation}.

%% ------------------------------------------------------------------------

\glent[ks~class]
\gllabel{ks-class}%
\gllabel{ks~classes}%
\index{ks class}%
\index{class!ks}%
%
An object that is a subclass of
\nobr{\textbf{\entlinknoex{standard-unit-class}}} that is used to represent a
\glref{KS}.

%% ------------------------------------------------------------------------

\glent[ksa~class]
\gllabel{ksa-class}%
\gllabel{ksa~classes}%
\index{ksa class}%
\index{class!ksa}%
%
An object that is a subclass of
\nobr{\textbf{\entlinknoex{standard-ksa-class}}} that is used to represent a
\glref{KSA}.

%% ------------------------------------------------------------------------

\glent[link]
\gllabel{links}%
\index{link}%
%
A bi-directional relationship between two \glref{unit~instances} represented
by a pair of pointers, one at each unit instance pointing to the other unit
instance.  GBBopen's link operators maintain the bi-directional consistency of
link pointers.

%% ------------------------------------------------------------------------

\glent[link~slot]
\gllabel{link-slot}%
\index{link slot}%
%
A \glref{slot} designated for the outgoing pointers of \glref{links}
associated with that \glref{slot}.

%% ------------------------------------------------------------------------

\glent[link-pointer~object] 
\gllabel{link-pointer-object}%
\gllabel{link-pointer~objects}%
\index{link-pointer object}%
\index{object!link-pointer}%
\indexit{link slot}%
\indexit{unit instance}%
\bfindexit{link-instance-of}%
%
An object that can be used as a pointer in a \nobr{\glref{link~slot}}.  A
link-pointer object must have a \nobr{\textbf{\entlink{link-instance-of}}}
method defined for it whose result returns the \glref{unit~instance} to be
used as the link pointer.

%% ------------------------------------------------------------------------

\glent[link-slot~place]
\gllabel{link-slot-place}%
\index{link slot!place}%
\index{place!link slot}%
%
A form which is suitable for use as a \glref{generalized~reference} to a
\glref{link~slot}.  Typical examples of link-slot-place forms include:
%
\W\supp\notpretop
\T\vspace{4pt}
\begin{example}
  (\var{slot-accessor unit-instance\/})
  (slot-value \var{unit-instance slot-name\/})
\end{example}
where:
\W\\~\\
\begin{args}{unit-instance}
\arg[slot-accessor] is a symbol specifying an accessor \glref{function} for
  a link slot
\arg[unit-instance] is a \glref{unit~instance}
\arg[slot-name] is a symbol naming a \glref{link~slot} in
\var{unit-instance\/} 
\end{args}

%% ------------------------------------------------------------------------

\glent[lock] 
\gllabel{locks}%
\index{lock}%
\index{lock!non-recursive}%
%
A mutual-exclusion object that allows multiple \glref{threads} to
synchronize activities or access to shared resources. A lock has two
states, unlocked or locked by a specific \glref{thread}. Once a lock
is held by a \glref{thread}, any other \glref{threads} attempting to
lock it will block. When the lock-holding \glref{thread} unlocks
(releases) the lock, one of the blocked \glref{threads} will acquire
(lock) it and proceed.  If the \glref{thread} that is holding the lock
attempts to re-acquire it, an error is signaled (see
\glref{recursive~lock}).

%% ------------------------------------------------------------------------

\glent[metaobject]
\index{metaobject}%
%
An \glref{instance} of a \glref{metaobject~class}.

%% ------------------------------------------------------------------------

\glent[metaobject~class]
\gllabel{metaobject-class}%
\index{class!metaobject}%
%
A \glref{class} object that is a subclass of exactly one of the following
classes: \code{class}, \code{slot-definition}, \code{generic-function},
\code{method}, and \code{method-combination}.

%% ------------------------------------------------------------------------

\glent[module]
\gllabel{modules}%
\index{module}%
%
A set of related files that form a component, library, or application.
Part of the \reflink{Module Manager Facility}{sec:module-manager}.

%% ------------------------------------------------------------------------

\glent[namestring]
\index{namestring}%
%
A string that represents a filename.

%% ------------------------------------------------------------------------

\glent[non-keyword~symbol]
\index{non-keyword symbol}%
\index{symbol!non-keyword}%
%
A symbol whose home package is not the \code{keyword} package.

%% ------------------------------------------------------------------------

\glent[obviated~KS~activation]
\gllabel{obviated~KSA}%
\gllabel{obviated~KSAs}%
\gllabel{obviated}%
\gllabel{obviation}%
\index{obviated knowledge-source activation}%
\index{KS activation!obviated}%
\index{KSA!obviated}%
%
An unexecuted \glref{KS~activation} that has been deemed unnecessary and will
therefore never be executed.

%% ------------------------------------------------------------------------

\glent[ordered~dimension]
\gllabel{ordered-dimension}%
\gllabel{ordered~dimensions}%
\index{ordered dimension}%
\index{dimension type!ordered}%
%
A \glref{dimension} of \code{:ordered} \glref{dimension~type} where
\code{:point}, \code{:interval}, and \code{:mixed} \glref{dimension~values}
are points or intervals on a continuous, real-number extent.

%% ------------------------------------------------------------------------

\glent[ordering~dimension]
\index{ordering dimension}%
%
The \glref{dimension} whose \glref{dimension~values} are used to order a
\glref{series-composite~dimension~value}.

%% ------------------------------------------------------------------------

\glent[ordered~queue]
\gllabel{ordered~queues}
\index{ordered queue}%
\index{queue!ordered}%
\bfindexit{ordered-queue}%
%
A doubly linked, ordered queue.  A GBBopen queue is headed by an object that
is a subclass of \nobr{\textbf{\entlinknoex{ordered-queue}}}.

%% ------------------------------------------------------------------------

\glent[package~designator]
\index{package!designator}%
%
A \glref{string~designator} (denoting a string that designates the name or
nickname of a package) or a package (denoting itself).

%% ------------------------------------------------------------------------

\glent[passive~socket]
\index{socket!passive}%
%
A socket that is used to accept a connection initiation to a specific service
port.

%% ------------------------------------------------------------------------

\glent[patch]
\gllabel{patches}%
\index{patch}%
%
A modification to the existing code of an application that is loaded after the
regular application code, either at startup or into an executing application.

%% ------------------------------------------------------------------------

\glent[path~expression]
\gllabel{path-expression}%
\index{path!expression}%
\index{path!space instances expression}%
%
A regular expression representing one or more \glref{space-instance~paths}.

%% ------------------------------------------------------------------------

\glent[pathname]
\index{pathname}%
%
A structured representation of the name of a file. A pathname has six
components: host, device, directory, name, type, and version.

%% ------------------------------------------------------------------------

\glent[pathname~designator]
\index{pathname!designator}%

A \glref{namestring} (denoting the corresponding \glref{pathname}), a stream
associated with a file (denoting the \glref{pathname} used to open the file),
or a \glref{pathname} (denoting itself).

%% ------------------------------------------------------------------------

\glent[pending~KS~activation]
\gllabel{pending~KSA}%
\gllabel{pending~KSAs}%
\index{pending knowledge-source activation}%
\index{KS activation!pending}%
\index{KSA!pending}%
A \glref{KS~activation} that has not been executed or \glref{obviated}.

%% ---------------------------------------------------------------------------

\glent[periodic~function]
\gllabel{periodic-function}%
\gllabel{periodic~functions}%

A \glref{function} of no arguments that is run repeatedly at a specified
interval, at a resolution as brief as supported by \code{sleep}. A separate
\glref{thread} is spawned to manage the periodic invocations of the specified
function.

A count can also be provided for the periodic function. When specified, this
value is decremented prior to each invocation of the function and, when it is
no longer positive, the periodic-function thread is terminated.

%% ------------------------------------------------------------------------

\glent[predicate~function]
\gllabel{predicate-function}%
\index{predicate function}%
\index{function!predicate}%
A \glref{function} that returns a \glref{generalized-boolean} value.

%% ------------------------------------------------------------------------

\glent[proper~list]
\index{proper list}%
\index{list!proper}%
A list terminated by the empty list. (The empty list is a proper list.)

%% ------------------------------------------------------------------------

\glent[property]
\index{property, of a property list}%
(of a \glref{property~list}) 1. A pair of elements in a \glref{property~list}
representing the name of a property and its associated value. 2. The value of
a property.

%% ------------------------------------------------------------------------

\glent[property~list]
\index{property list}%
\index{list!property list}%
A list containing an even number of elements that represent alternating names
(sometimes called indicators or keys) and their associated values.

%% ------------------------------------------------------------------------

\glent[pseudo-probability]

A discretized \code{\glref{fixnum}} representation for probability values that
maps probability values in the range [0.0..1.0] to integers in the range
[0..1000].

%% ------------------------------------------------------------------------

\glent[pseudo-probability-ln]

A discretized \code{\glref{fixnum}} representation for the natural logarithm
of a \glref{pseudo-probability} value.  Pseudo-probability-ln values range
from [-6907756..0].

%% ------------------------------------------------------------------------

\glent[queue]
\index{queue}%
\index{doubly-linked queue}%
A doubly linked queue.  A GBBopen queue is headed by an object that is a
subclass of \textbf{\entlinknoex{queue}}.  GBBopen queues that maintain a
sorted ordering of \glref{queue~elements} are provided by
\glref{ordered~queues}.

%% ------------------------------------------------------------------------

\glent[queue~element]
\gllabel{queue-element}%
\gllabel{queue~elements}%
\index{queue element}%
An object that is a subclass of \nobr{\textbf{\entlinknoex{queue-element}}}.

%% ------------------------------------------------------------------------

\glent[quiescence]
\index{agenda Shell!quiescence}%
\index{control shell!quiescence}%
A control-shell state when no more \glref{executable~KSAs} are in the queue of
\glref{pending~KSAs}.

%% ------------------------------------------------------------------------

\glent[rating]
\index{rating!of a KSA}%
An integer between -32768 and 32767 inclusive, used by the Agenda Shell to
order \glref{pending~KSAs} (see \textbf{\entlinknoex{rating}}).

%% ------------------------------------------------------------------------

\glent[recursive~lock] 
\gllabel{recursive-lock}%
\gllabel{recursive~locks}%
\index{recursive lock}%
\index{lock!recursive}%
%
A mutual-exclusion object that allows multiple \glref{threads} to synchronize
activities or access to shared resources. A recursive lock has two states,
unlocked or locked by a specific \glref{thread}. Once a recursive lock is held
by a \glref{thread}, any other \glref{threads} attempting to lock it will
block. When the lock-holding \glref{thread} unlocks (releases) the recursive
lock, one of the blocked \glref{threads} will acquire (lock) it and proceed.
If the \glref{thread} that is holding the recursive lock attempts to
re-acquire it, that \glref{thread} is allowed to proceed as if it had acquired
the lock (without error or blocking, see \glref{lock}).

%% ------------------------------------------------------------------------

\glent[relative~directory]
\gllabel{relative-directory}%
\gllabel{relative~directories}%
\index{relative directory}%
\index{directory!relative}%
%
A directory defined in relation to another directory definition.
Part of the \reflink{Module Manager Facility}{sec:module-manager}.

%% ------------------------------------------------------------------------

\glent[REPL~command] 
\gllabel{command}%
\gllabel{commands}%
\gllabel{keyword~command}%
\gllabel{keyword-command}%
\gllabel{keyword~commands}%
\gllabel{REPL~commands}%
\index{REPL command}%
\index{command!REPL}%
%
A \glref{keyword} command that can be entered at the top level read-eval-print
loop (REPL) in your Common Lisp environment.  REPL commands (some with
arguments) provide convenient shortcuts for often-used operations.

%% ------------------------------------------------------------------------

\glent[retrieval~pattern]
\gllabel{retrieval~patterns}%
\index{retrieval!pattern}%
\index{pattern!retrieval}%
\bfindexit{filter-instances}%
\bfindexit{find-instances}%
\bfindexit{map-instances-on-space-instances}%
%
A list argument to \nobr{\textbf{\entlinknoex{filter-instances}}},
\nobr{\textbf{\entlinknoex{find-instances}}}, and
\nobr{\textbf{\entlinknoex{map-instances-on-space-instances}}} specifying the
\glref{dimension~value} requirements for selecting \glref{unit~instances} to
be returned.

%% ------------------------------------------------------------------------

\glent[required~module]
\gllabel{required-module}%
\gllabel{required~modules}%
\index{module}%
%
A sequence of \glref{modules} that must be compiled (if needed) and loaded, in
order, before the requiring \glref{module} is compiled (if needed) and loaded.
Part of the \reflink{Module Manager Facility}{sec:module-manager}.

%% ------------------------------------------------------------------------

\glent[root~directory]
\gllabel{root-directory}%
\gllabel{root}%
\gllabel{Root}%
\gllabel{Root~directories}%
\index{root directory}%
\index{directory!root}%
%
A fixed anchor directory for a tree of relative directory definitions.
Part of the \reflink{Module Manager Facility}{sec:module-manager}.

%% ---------------------------------------------------------------------------

\glent[scheduled~function]
\gllabel{scheduled~functions}%
\gllabel{Scheduled~functions}%
\gllabel{scheduled-function}%
%
An object that contains a \glref{function} that may be scheduled to run at an
absolute or relative time. When that specified time arrives, the function
is invoked with a single argument: the scheduled-function object.

A repeat interval can also be specified for the scheduled
function. When specified, this value is used whenever the function is
invoked to schedule the function again at a new time relative to the
current invocation.

Scheduled functions can be scheduled to a resolution of one second.  Periodic
function invocations at brief time intervals are provided by
\glref{periodic~functions}.

%% ------------------------------------------------------------------------

\glent[series-composite~dimension~value]
\gllabel{series-composite~dimension~values}%
\index{composite dimension!value, series}% 
\index{series-composite dimension!value}% 
%
A \glref{dimension~value} that is a series of \glref{dimension~values} that
are ordered by the \glref{dimension~values} of another \glref{dimension} (the
\glref{series-composite~ordering~dimension}).

%% ------------------------------------------------------------------------

\glent[series-composite~ordering~dimension]
\index{composite dimension!ordering dimension, of a series-composite 
  dimension}% 
\index{series-composite dimension!ordering dimension}% 
\index{ordering dimension, of a series-composite dimension}% 
%
The \glref{dimension} whose values are used to order a
\glref{series-composite~dimension~value}.

%% ------------------------------------------------------------------------

\glent[sequence-composite~dimension~value]
\index{composite dimension!value, sequence}% 
\index{sequence-composite dimension!value}% 
%
A dimension value that is a sequence of \glref{dimension~values}.

%% ------------------------------------------------------------------------

\glent[set-composite~dimension~value]
\index{composite dimension!value, set}% 
\index{set-composite dimension!value}% 
%
A dimension value that is a set of \glref{dimension~values}.

%% ------------------------------------------------------------------------

\glent[slot]
\index{slot}%
\index{object!slot}%
%
A component of an object that can store a value.

%% ------------------------------------------------------------------------

\glent[space]
\index{space!class}%
\index{space!instance}%
\index{instance!space instance}%
%
Pertaining to a \glref{space~class} or \glref{space~instance}.

%% ------------------------------------------------------------------------

\glent[space~class]
\gllabel{space~classes}%
\index{class!space}%
\index{space!class}%
\bfindexit{standard-space-class}%
%
An object that is a subclass of \nobr{\textbf{\entlinknoex{standard-space-class}}}.

%% ------------------------------------------------------------------------

\glent[space~instance]
\gllabel{space-instance}%
\gllabel{space~instances}%
\index{space!instance}%
\index{instance!space instance}%
\bfindexit{standard-space-instance}%
%
An object whose class is a subclass of
\nobr{\textbf{\entlinknoex{standard-space-instance}}}.  A space instance is also a
\glref{unit~instance}, but space instances are normally described by their
more specific categorization.

%% ------------------------------------------------------------------------

\glent[space-instance~path] 
\gllabel{space-instance~paths}%
\index{space instance!path}%
\index{path!space instance}%
%
A complete list of space-instance names, starting with the most distant
indirect parent space-instance name, that uniquely identifies a
\glref{space~instance} in the \glref{blackboard~repository}.

%% ------------------------------------------------------------------------

\glent[standard-gbbopen-instance]
\bfindexit{standard-gbbopen-instance}%
%
An object whose class is a subclass of 
\nobr{\textbf{\entlinknoex{standard-gbbopen-instance}}}. It is a superclass of
\nobr{\textbf{\entlink{standard-event-instance}}} and
\nobr{\textbf{\entlink{standard-unit-instance}}}.

%% ------------------------------------------------------------------------

\glent[storage~specification]
\index{storage specification}%
\index{space instance!storage specification}%
%
A specification of how \glref{unit~instances} are to be stored on a
\glref{space~instance}.

%% ------------------------------------------------------------------------

\glent[string~designator]
\index{string designator}%
%
An object that denotes a string. One of: a character (denoting a
singleton string that has the character as its only element), a symbol
(denoting the string that is its name), or a string (denoting itself).

%% ------------------------------------------------------------------------

\glent[subclasses]
\index{unit class!subclasses}%
\index{class!subclasses}%
%
The classes that inherit from a \glref{class}.

%% ------------------------------------------------------------------------

\glent[subevents]
\index{event class!subevents}%
%
The classes that inherit from an \glref{event~class}.

%% ------------------------------------------------------------------------

\glent[system~name] \gllabel{system}%
\index{system-name, keyword symbol}%
%
A \glref{keyword} associating a set of REPL commands, directory
definitions, and module definitions with a specific library or application.

%% ------------------------------------------------------------------------

\glent[thread] 
\gllabel{threads}%
\index{thread}%
%
A thread in a multi-threaded Common Lisp implementation or a Lisp process in a
Common Lisp that provides multiprocessing.

%% ------------------------------------------------------------------------

\glent[time~zone]
\index{time zone}%
%
A rational number between -24 (inclusive) and 24 (inclusive) that represents a
time zone as a number of hours offset from Greenwich Mean Time. A non-integral
time zone must be a multiple of 
%
\W1/3600.  
\T$\frac{1}{3600}$.

%% ------------------------------------------------------------------------

\glent[unit]
\index{unit!class}%
\index{unit!instance}%
\index{class!unit}%
\index{instance!unit}%
%
Pertaining to a \glref{unit~class} or \glref{unit~instance}.

%% ------------------------------------------------------------------------

\glent[unit~class]
\gllabel{unit-class}%
\gllabel{Unit~classes}%
\gllabel{unit~classes}%
\index{unit!class}%
\index{class!unit}%
\bfindexit{standard-unit-class}%
%
An object that is a subclass of \nobr{\textbf{\entlinknoex{standard-unit-class}}}.

%% ------------------------------------------------------------------------

\glent[unit~instance] 
\gllabel{unit~instances}%
\gllabel{Unit~instances}%
\gllabel{unit-instance}%
\gllabel{instance}%
\gllabel{instances}%
\index{unit!instance}%
\index{instance!unit}%
\bfindexit{standard-unit-instance}%
%
An object whose class is a subclass of
\nobr{\textbf{\entlinknoex{standard-unit-instance}}}. A \glref{space~instance} is
also a unit instance, but \glref{space~instances} are normally described by
their more specific categorization.

%% ------------------------------------------------------------------------

\glent[universal~time]
\index{universal time}%
%
A non-negative integer number of seconds measured from the beginning of the
year 1900 (ignoring leap seconds).

%% ------------------------------------------------------------------------

\glent[variable~symbol]
\index{variable symbol}%
%
A symbol that can accept a binding.

\end{glossary-list}

%% ========================================================================
