%% -*- Mode:TeX; Fonts:(hl12fb) -*-
%% *-* File: /usr/local/gbbopen/doc-source/portable-threads-entities.tex *-*
%% *-* Last-Edit: Tue Oct 27 22:06:20 2009; Edited-By: cork *-*
%% *-* Machine: cyclone.cs.umass.edu *-*

\index{module!:portable-threads@\code{:portable-threads}}%
\index{:portable-threads@\code{:portable-threads} module}%
GBBopen's Portable Threads provides a uniform interface to commonly used
\glref{thread} (multiprocessing) entities.  Wherever possible, these entities
do something reasonable in Common Lisp implementations that do not provide
threads. However, entities that make no sense without threads signal errors in
non-threaded implementations (as noted with each entity).  The \glref{feature}
\nobr{\code{:threads-not-available}} is added on Common Lisp implementations
without thread support, and the \glref{feature}
\nobr{\code{:with-timeout-not-available}} is added on implementations that do
not support \nobr{\textbf{\entlink{with-timeout}}}.

Portable Threads entities are provided by the \nobr{\code{:portable-threads}}
module in GBBopen.  Stand-alone use of the Portable Threads interface is also
easy, requiring only the
\xsitelink{\nobr{\code{portable-threads.lisp}}}{http://gbbopen.org/svn/GBBopen/trunk/source/tools/portable-threads.lisp}
file.

\psubpar{Threads and Processes}

Common Lisp implementations that provide multiprocessing capabilities use one
of two approaches:
\begin{tightitemize}
\item \textit{Application-level threads\/} (also called ``Lisp processes'') which are
  created, deleted, and scheduled internally by the Common Lisp implementation
\item \textit{Operating-system threads\/} (or ``native threads'') which are
  lightweight, operating-system threads that are created, deleted, and
  scheduled by the operating system
\end{tightitemize}

There are advantages and complexities associated with each approach, and the
Portable Threads Interface is designed to provide a uniform abstraction over
them that can be used to code applications that perform consistently and
efficiently on any supported Common Lisp implementation.

\psubpar{Locks}

Common Lisp implementations provide differing semantics for the behavior of
mutual-exclusion locks that are acquired recursively by the same
\glref{thread}: some always allow recursive use, others provide special
``recursive'' lock objects in addition to non-recursive locks, and still
others allow recursive use to be specified at the time that a lock is being
acquired.  To enable behavioral consistency in all Common Lisp
implementations, the \nobr{\code{:portable-threads}} interface module provides
(non-recursive) \glref{locks} and \glref{recursive~locks} and a single
acquisition form, \nobr{\textbf{\entlink{with-lock-held}}}, that behaves
appropriately for each lock type.

\psubpar{Condition Variables}

POSIX-style \glref{condition~variables} provide an atomic means for a
\glref{thread} to release a lock that it holds and go to sleep until it is
awakened by another thread.  Once awakened, the lock that it was holding is
reacquired atomically before the thread is allowed to do anything else.

A condition variable must always be associated with a \glref{lock} (or
\glref{recursive~lock}) in order to avoid a race condition created when one
thread signals a condition while another thread is preparing to wait on it.
In this situation, the second thread would be perpetually waiting for the
signal that has already been sent.  In the POSIX model, there is no explicit
link between the lock used to control access to the condition variable and the
condition variable.  The Portable Threads Interface makes this association
explicit by bundling the lock with the
\nobr{\textbf{\entlink{condition-variable}}} CLOS object instance and allowing
the \nobr{\textbf{\entlink{condition-variable}}} object to be used directly in
lock entities.

\psubpar{Hibernation}

Sometimes it is desirable to put a thread to sleep (perhaps for a long time) until some event has occurred.  The Portable Threads Interface provides two entities that make this situation easy to code: \nobr{\textbf{\entlink{hibernate-thread}}} and \nobr{\textbf{\entlink{awaken-thread}}}.  Thread hibernation can only be performed by the thread on itself, eliminating issues of a thread being hibernated at an undesirable time.  Note that there is the potential for a hibernate/awaken race condition if a thread hibernates itself again soon after being awakened (when a second \nobr{\textbf{\entlink{awaken-thread}}} intended for the original hiberation is applied to the second hibernation rather than being ignored because the target thread is not hibernating).  Using a \nobr{\textbf{\entlink{condition-variable}}} is preferable in this situation.

When a thread is hibernating, it remains available to respond to \nobr{\textbf{\entlink{run-in-thread}}} and \nobr{\textbf{\entlink{symbol-value-in-thread}}} operations as well as to be awakened by a dynamically surrounding \nobr{\textbf{\entlink{with-timeout}}}.

\psubpar{What about Process Wait?}

Thread coordination functions, such as \nobr{\code{process-wait}}, are
expensive to implement with operating-system threads.  Such functions stop the
executing thread until a Common Lisp \glref{predicate~function} returns a true
value.  With application-level threads, the Lisp-based scheduler evaluates the
\glref{predicate~function} periodically when looking for other threads that
can be run.  With operating-system threads, however, thread scheduling is
performed by the operating system and evaluating a Common Lisp
\glref{predicate~function} requires complex and expensive interaction between
the operating-system thread scheduling and the Common Lisp implementation.
Given this cost and complexity, many Common Lisp implementations that use
operating-system threads have elected not to provide
\nobr{\code{process-wait}}-style coordination functions, and this issue
extends to the Portable Threads Interface as well.

Fortunately, most uses of \nobr{\code{process-wait}} can be replaced by a
different strategy that relies on the producer of a change that would affect
the \nobr{\code{process-wait}} \glref{predicate~function} to signal the event
rather than having the consumers of the change use predicate functions to poll
for it.  Condition variables, the Portable Threads
\nobr{\textbf{\entlink{hibernate-thread}}} and
\nobr{\textbf{\entlink{awaken-thread}}} mechanism, or blocking I/O functions
cover most of the typical uses of \nobr{\code{process-wait}}.

\psubpar{Scheduled Functions}

A \glref{scheduled~function} is an object that contains a \glref{function} to
be run at a specified time. When that specified time arrives, the
\glref{function} is invoked with a single argument: the
\glref{scheduled~function} object. A repeat interval (in seconds) can also be
specified for the \glref{scheduled~function}. This value is used whenever the
\glref{scheduled~function} is invoked to schedule itself again at a new time
relative to the current invocation.  \glref{Scheduled~functions} can be
scheduled to a resolution of one second.

\glref{Scheduled~functions} are scheduled and invoked by a separate
\nobr{\code{"Scheduled-Function Scheduler"}} \glref{thread}.  Unless the run
time of the invoked \glref{function} is brief, the \glref{function} should
spawn a new \glref{thread} in which to perform its activities so as to avoid
delaying the invocation of a subsequent \glref{scheduled~function}.

\psubpar{Periodic Functions} 

\codeindexit{sleep}%
%
A \glref{periodic~function} is a \glref{function} to be run repeatedly at a
specified interval.  Unlike \glref{scheduled~functions}, which can be
scheduled only to a resolution of one second, a \glref{periodic~function} can
be repeated at intervals as brief as is supported by the \nobr{\code{sleep}}
function of the Common Lisp implementation.  A \glref{periodic~function} is
scheduled and executed in its own \glref{thread}.  As with
\glref{scheduled~functions}, however, \glref{function} should spawn a new
\glref{thread} in which to perform its activities, unless its run time is
brief.

\W\entities
\T\psubpar{Portable Threads entities}
\T Descriptions of the Portable Threads entities follow.
\T\clearpage

%% ------------------------------------------------------------------------

\begin{functiondoc}[periodic-function-verbose-var]{Variable}%
  {*periodic-function-verbose*}{}%

\fnsyntax

\fnpurpose Controls whether initiation and termination of
\glref{periodic-function} \glref{threads} are printed as comments.

\fnpackage \code{:portable-threads}

\fnmodule \code{:portable-threads}

\fnvaluetype A \glref{generalized~boolean}

\fninitialvalue \nil

\fndescription The value of \nobr{\textbf{*periodic-function-verbose*}} can be
changed globally to display the management of \glref{periodic~functions}.

\begin{alsos}{spawn-periodic-function}
\also[kill-periodic-function]
\also[spawn-periodic-function]
\end{alsos}

\bfindexit{spawn-periodic-function}%
\fnexample 
Schedule a simple \glref{periodic~function} with verbose printing enabled:
%
\W\supp
\begin{example}
  > (setf *periodic-function-verbose* 't)
  t
  > (\entlink{spawn-periodic-function} #'(lambda () (print "Hello!")) 0.1 
      :name 'hello
      :count 2)
  ;; Spawning periodic-function thread for hello...
  #<thread Periodic Function hello>
  >
  "Hello!" 
  "Hello!" 
  ;; Exiting periodic-function thread hello
  >
\end{example}

\end{functiondoc}

%% ------------------------------------------------------------------------

\begin{functiondoc}[schedule-function-verbose-var]{Variable}%
  {*schedule-function-verbose*}{}%

\fnsyntax

\fnpurpose Controls whether scheduling changes made to
\glref{scheduled~functions} are printed as comments. 

\fnpackage \code{:portable-threads}

\fnmodule \code{:portable-threads}

\fnvaluetype A \glref{generalized~boolean}

\fninitialvalue \nil

\fndescription The value of \nobr{\textbf{*schedule-function-verbose*}} can be
changed globally to display the activities of the \glref{scheduled~function}
scheduler.

\begin{alsos}{schedule-function-relative}
\also[schedule-function]
\also[schedule-function-relative]
\also[unschedule-function]
\end{alsos}

\bfindexit{encode-time-of-day}%
\fnexample Change the invocation time of \glref{scheduled~function}
\nobr{\code{quitting-time}} from 5pm to 5:30pm with verbose printing enabled:
%
\W\supp
\begin{example}
  > (setf *schedule-function-verbose* 't)
  t
  > (schedule-function 'quitting-time (\entlink{encode-time-of-day} 17 30 0)
      :repeat-interval #.(* 24 60 60))
  ;; Unscheduling #<scheduled-function quitting-time [17:00:00]>...
  ;; Scheduling #<scheduled-function quitting-time [17:30:00]> 
  ;; as the next scheduled-function...
  >
\end{example}

\end{functiondoc}

%% ------------------------------------------------------------------------

\begin{functiondoc}{Function}{all-scheduled-functions}{\noargs{} 
    \returns{} \var{list-of-scheduled-functions\/}}
\index{scheduled functions!obtaining all}%
\index{function!scheduled, obtaining all}%

\fnsyntax

\fnpurpose Return a list of all \glref{scheduled~functions} that are
currently scheduled.

\fnpackage \code{:portable-threads}

\fnmodule \code{:portable-threads}

\fnargs
\begin{args}{list-of-scheduled-functions}
\arg[list-of-scheduled-functions] A proper list
\end{args}

\fnreturns A list of \nobr{\code{scheduled-function}} objects. 

\begin{alsos}{schedule-function-relative}
\also[make-scheduled-function]
\also[schedule-function]
\also[schedule-function-relative]
\also[unschedule-function]
\end{alsos}

\fnexample
%
\W\supp
\begin{example}
  > (all-threads)
  (#<thread Listener 1>)
  >
\end{example}

\fnnotes On Common Lisp implementations without threads, \nil{} is returned.

The returned list of \glref{scheduled~functions} should not be
destructively altered.

\end{functiondoc}

%% ------------------------------------------------------------------------

\begin{functiondoc}{Function}{all-threads}{\noargs{} 
    \returns{} \var{list-of-threads\/}}
\index{thread!obtaining all}%

\fnsyntax

\fnpurpose Return a list of all known \glref{threads}.

\fnpackage \code{:portable-threads}

\fnmodule \code{:portable-threads}

\fnargs
\begin{args}{list-of-threads}
\arg[list-of-threads] A proper list
\end{args}

\fnreturns A list of objects representing the \glref{threads}.

\fndescription The returned list of \glref{threads} is accurate only at the
precise instant the \nobr{\textbf{all-threads}} function is called.  New
threads may be created or existing threads killed at any time, so the returned
list is always potentially outdated.

\begin{alsos}{thread-alive-p}
\also[current-thread]
\also[thread-alive-p]
\also[threadp]
\also[spawn-form]
\also[spawn-thread]
\end{alsos}

\fnexample
%
\W\supp
\begin{example}
  > (all-threads)
  (#<thread Listener 1>)
  >
\end{example}

\fnnotes On Common Lisp implementations without threads, \nil{} is returned.

The returned list of \glref{threads} should not be destructively
altered.

\end{functiondoc}

%% ------------------------------------------------------------------------

\begin{functiondoc}{Macro}{as-atomic-operation}%
  {\var{form\/}\superstar{} 
    \returns{} \var{primary-value\/}}

\fnsyntax

\fnpurpose Execute \var{forms\/} as an atomic operation.

\fnpackage \code{:portable-threads}

\fnmodule \code{:portable-threads}

\fnargs
\begin{args}{form}
\arg[primary-value] The first value returned by evaluating the last 
\var{form}
\end{args}

\fnreturns The primary value returned by evaluating the last
\nobr{\var{form}}.

\fndescription This macro provides atomicity in the following entities (when
the Common Lisp implementation does not support them directly):
%
\nobr{\textbf{\entlink{atomic-decf}}},
\nobr{\textbf{\entlink{atomic-decf\&}}},
\nobr{\textbf{\entlink{atomic-delete}}},
\nobr{\textbf{\entlink{atomic-flush}}}, 
\nobr{\textbf{\entlink{atomic-incf}}},
\nobr{\textbf{\entlink{atomic-incf\&}}},
\nobr{\textbf{\entlink{atomic-push}}},
\nobr{\textbf{\entlink{atomic-pushnew}}}, and
\nobr{\textbf{\entlink{atomic-pop}}}.  
%
It is intended only for implementing very brief atomic operations and should
not be used for long computations or computations that wait or block.

Note that \nobr{\textbf{as-atomic-operation}} is only guaranteed to return a
single value, not multiple values.

\begin{alsos}{browse-hyperdoc}
\also[atomic-decf]
\also[atomic-decf\&]
\also[atomic-delete]
\also[atomic-flush]
\also[atomic-incf]
\also[atomic-incf\&]
\also[atomic-push]
\also[atomic-pushnew]
\also[atomic-pop]
\end{alsos}

\fnexample
Define an atomic \nobr{\textbf{\entlink{nsorted-insert}}}:
%
\W\supp
\begin{example}
  (defun atomic-nsorted-insert (\&rest args)
    (declare (dynamic-extent args))
    (as-atomic-operation (apply #'\entlink{nsorted-insert} args)))
\end{example}

\end{functiondoc}

%% ------------------------------------------------------------------------

\begin{functiondoc}{Macro}{atomic-decf}{\var{place\/} [\var{delta-form\/}]
      \returns{} \var{new-place-value\/}}
\index{atomic operations!decf@\textbf{decf}}%

\fnsyntax \fnpurpose Decrement the value stored in \nobr{\var{place\/}} as an
\glref{atomic~operation}.

\fnpackage \code{:portable-threads}

\fnmodule \code{:portable-threads}

\fnargs
\begin{args}{place}
\arg[place] A \glref{form} which is suitable for use as a
\glref{generalized~reference} 
\arg[delta-form] A \glref{form} that is evaluated to produce a delta value
(default is 1)
\arg[new-place-value] A number
\end{args}

\fnreturns The new value of \var{place}. 

\begin{alsos}{as-atomic-operation}
\also[as-atomic-operation]
\also[atomic-decf\&]
\also[atomic-delete]
\also[atomic-flush]
\also[atomic-incf]
\also[atomic-incf\&]
\also[atomic-pop]
\also[atomic-pushnew]
\end{alsos}

\fnexamples
%
\W\supp
\begin{example}
  > x
  5
  > (atomic-decf x)
  4
  > (atomic-decf x 1.5)
  2.5
  >
\end{example}

\end{functiondoc}

%% ------------------------------------------------------------------------

\begin{functiondoc}[atomic-decf-amp]{Macro}{atomic-decf\&}{\var{place\/}
    [\var{delta-form\/}]
    \returns{} \var{new-place-value\/}}
\index{atomic operations!decf\&@\textbf{decf\&}}%

\fnsyntax \fnpurpose Decrement the \glref{fixnum} value stored in
\nobr{\var{place\/}} as an \glref{atomic~operation}.

\fnpackage \code{:portable-threads}

\fnmodule \code{:portable-threads}

\fnargs
\begin{args}{place}
\arg[place] A \glref{form} which is suitable for use as a
\glref{generalized~reference} containing a fixnum value
\arg[delta-form] A \glref{form} that is evaluated to produce a fixnum delta
value (default is 1)
\arg[new-place-value] A fixnum
\end{args}

\fnreturns The new fixnum value of \var{place}. 

\begin{alsos}{as-atomic-operation}
\also[as-atomic-operation]
\also[atomic-decf]
\also[atomic-delete]
\also[atomic-flush]
\also[atomic-incf]
\also[atomic-incf\&]
\also[atomic-pop]
\also[atomic-pushnew]
\end{alsos}

\fnexamples
%
\W\supp
\begin{example}
  > x
  5
  > (atomic-decf\& x)
  4
  > (atomic-decf\& x 2)
  2
  >
\end{example}

\end{functiondoc}

%% ------------------------------------------------------------------------

\begin{functiondoc}{Macro}{atomic-delete}{\var{item place\/}
    \code{\&key} \var{from-end test test-not start end count key\/}
    \returns{} \var{sequence\/}}
\index{atomic operations!delete@\textbf{delete}}%

\fnsyntax

\fnpurpose As an \glref{atomic~operation}, set \nobr{\var{place\/}} to the
sequence in \nobr{\var{place\/}} from which the elements that satisfy the
\nobr{\var{test\/}} have been removed.

\fnpackage \code{:portable-threads}

\fnmodule \code{:portable-threads}

\fnargs
\begin{args}{from-end}
\arg[item] An object
\arg[place] A \glref{form} which is suitable for use as a
\glref{generalized~reference} that contains a proper sequence
\arg[from-end] A \glref{generalized~boolean} (default is \nil)
\arg[test] A \glref{function} of two arguments that returns a
\glref{generalized~boolean} (default is \nobr{\code{\#'eql}}) 
\arg[test-not] A \glref{function} of two arguments that returns a
\glref{generalized~boolean} (use of \nobr{\code{:test-not}} is deprecated)
\arg[start] Starting index into \nobr{\var{sequence\/}} (default is \code{0})
\arg[end] Ending index into \nobr{\var{sequence\/}} (default is \nil, meaning
end of \nobr{\var{sequence\/}})
\arg[count] An integer or \nil{} (default is \nil)
\arg[key] A \glref{function} of one argument, or \nil{} (default is \nil)
\arg[sequence] A sequence
\end{args}

\fnreturns The sequence in \var{place\/} from which the elements that
satisfy the test have been removed.

\fndescription Replaces \var{place\/} with the sequence in
\nobr{\var{place\/}} from which elements that satisfy the \nobr{\var{test\/}}
have been deleted.  The supplied \nobr{\var{place\/}} sequence may be modified
in constructing the result; however, modification of the sequence itself is
not guaranteed.

Specifying a \nobr{\var{from-end\/}} value of true matters only when
the \nobr{\var{count\/}} is provided, and in that case only the rightmost
\nobr{\var{count\/}} elements satisfying the \nobr{\var{test\/}} are deleted.

\begin{alsos}{as-atomic-operation}
\also[as-atomic-operation]
\also[atomic-flush]
\also[atomic-pop]
\also[atomic-push]
\also[atomic-pushnew]
\also[counted-delete]
\end{alsos}

\fnexample
%
\W\supp
\begin{example}
  > list
  (1 2 3)
  > (atomic-delete 2 list)
  (2 3)
  > list
  (2 3)
  >
\end{example}

\end{functiondoc}

%% ------------------------------------------------------------------------

\begin{functiondoc}{Macro}{atomic-flush}{\var{place\/} \returns{}
    \var{old-place-value\/}}
\index{atomic operations!flush}%

\fnsyntax

\fnpurpose As an \glref{atomic~operation}, set the value of
\nobr{\var{place\/}} to \nil, and return the value \nobr{\var{place\/}} had
prior to being set.

\fnpackage \code{:portable-threads}

\fnmodule \code{:portable-threads}

\fnargs
\begin{args}{old-place-value}
\arg[place] A \glref{form} which is suitable for use as a
\glref{generalized~reference} 
\arg[old-place-value] An object
\end{args}

\fnreturns The \var{place\/} value prior to being set to \nil.

\begin{alsos}{as-atomic-operation}
\also[as-atomic-operation]
\also[atomic-delete]
\also[atomic-pop]
\also[atomic-push]
\also[atomic-pushnew]
\end{alsos}

\fnexample
%
\W\supp
\begin{example}
  > list
  (1 2 3)
  > (atomic-flush list)
  (1 2 3)
  > list
  nil
  >
\end{example}

\end{functiondoc}

%% ------------------------------------------------------------------------

\begin{functiondoc}{Macro}{atomic-incf}{\var{place\/} [\var{delta-form\/}]
      \returns{} \var{new-place-value\/}}
\index{atomic operations!incf@\textbf{incf}}%

\fnsyntax \fnpurpose Increment the value stored in \nobr{\var{place\/}} as an
\glref{atomic~operation}.

\fnpackage \code{:portable-threads}

\fnmodule \code{:portable-threads}

\fnargs
\begin{args}{new-place-value}
\arg[place] A \glref{form} which is suitable for use as a
\glref{generalized~reference} 
\arg[delta-form] A \glref{form} that is evaluated to produce a delta value
(default is 1)
\arg[new-place-value] A number
\end{args}

\fnreturns The new value of \var{place}. 

\begin{alsos}{as-atomic-operation}
\also[as-atomic-operation]
\also[atomic-decf]
\also[atomic-decf\&]
\also[atomic-delete]
\also[atomic-flush]
\also[atomic-incf\&]
\also[atomic-pop]
\also[atomic-pushnew]
\end{alsos}

\fnexamples
%
\W\supp
\begin{example}
  > x
  2
  > (atomic-incf x)
  3
  > (atomic-incf x 1.5)
  4.5
  >
\end{example}

\end{functiondoc}

%% ------------------------------------------------------------------------

\begin{functiondoc}[atomic-incf-amp]{Macro}{atomic-incf\&}{\var{place\/} 
    [\var{delta-form\/}]
    \returns{} \var{new-place-value\/}}
\index{atomic operations!incf\&@\textbf{incf\&}}%

\fnsyntax \fnpurpose Increment the \glref{fixnum} value stored in
\nobr{\var{place\/}} as an \glref{atomic~operation}.

\fnpackage \code{:portable-threads}

\fnmodule \code{:portable-threads}

\fnargs
\begin{args}{new-place-value}
\arg[place] A \glref{form} which is suitable for use as a
\glref{generalized~reference} containing a fixnum value
\arg[delta-form] A \glref{form} that is evaluated to produce a fixnum delta
value (default is 1)
\arg[new-place-value] A fixnum
\end{args}

\fnreturns The new fixnum value of \var{place}. 

\begin{alsos}{as-atomic-operation}
\also[as-atomic-operation]
\also[atomic-decf]
\also[atomic-decf\&]
\also[atomic-delete]
\also[atomic-flush]
\also[atomic-incf]
\also[atomic-pop]
\also[atomic-pushnew]
\end{alsos}

\fnexamples
%
\W\supp
\begin{example}
  > x
  2
  > (atomic-incf\& x)
  3
  > (atomic-incf\& x 2)
  5
  >
\end{example}

\end{functiondoc}

%% ------------------------------------------------------------------------

\begin{functiondoc}{Macro}{atomic-pop}{\var{place\/} \returns{}
    \var{element\/}}
\index{atomic operations!pop@\textbf{pop}}%

\fnsyntax

\fnpurpose As an \glref{atomic~operation}, remove the first element
from the list stored in \nobr{\var{place}}, store the updated list in
\nobr{\var{place}}, and return the removed first element.

\fnpackage \code{:portable-threads}

\fnmodule \code{:portable-threads}

\fnargs
\begin{args}{element}
\arg[place] A \glref{form} which is suitable for use as a
  \glref{generalized~reference} that contains a \glref{proper~list} or
   a \glref{dotted~list}
\arg[element] An object
\end{args}

\fnreturns The first element (the \glref{car}) of the list stored in
\nobr{\var{place}}.

\begin{alsos}{as-atomic-operation}
\also[as-atomic-operation]
\also[atomic-delete]
\also[atomic-flush]
\also[atomic-push]
\also[atomic-pushnew]
\end{alsos}

\fnexample
%
\W\supp
\begin{example}
  > list
  (1 2 3)
  > (atomic-pop list)
  1
  > list
  (2 3)
  >
\end{example}

\end{functiondoc}

%% ------------------------------------------------------------------------

\begin{functiondoc}{Macro}{atomic-push}{\var{item place\/} \returns{}
    \var{new-place-value\/}}
\index{atomic operations!push@\textbf{push}}%

\fnsyntax

\fnpurpose As an \glref{atomic~operation}, prepend \var{item\/} to the
list stored in \nobr{\var{place\/}} and store the updated list in
\nobr{\var{place}}.

\fnpackage \code{:portable-threads}

\fnmodule \code{:portable-threads}

\fnargs
\begin{args}{new-place-value}
\arg[item] An object
\arg[place] A \glref{form} which is suitable for use as a
\glref{generalized~reference} 
\arg[new-place-value] A proper list
\end{args}

\fnreturns The new value of \var{place}. 

\begin{alsos}{as-atomic-operation}
\also[as-atomic-operation]
\also[atomic-delete]
\also[atomic-flush]
\also[atomic-pop]
\also[atomic-pushnew]
\end{alsos}

\fnexample
%
\W\supp
\begin{example}
  > list
  (1 2 3)
  > (atomic-push 10 list)
  (10 1 2 3)
  >
\end{example}

\end{functiondoc}

%% ------------------------------------------------------------------------

\begin{functiondoc}{Macro}{atomic-pushnew}{\var{item place\/} 
    \code{\&key} \var{key test test-not\/}
    \returns{} \var{new-place-value\/}}
\index{atomic operations!pushnew@\textbf{pushnew}}%

\fnsyntax

\fnpurpose As an \glref{atomic~operation}, when \var{item\/} is not
the same as any element in the list stored in \nobr{\var{place}}, prepend
\var{item\/} to the list and store the updated list in \nobr{\var{place}}.

\fnpackage \code{:portable-threads}

\fnmodule \code{:portable-threads}

\fnargs
\begin{args}{new-place-value}
\arg[item] An object
\arg[place] A \glref{form} which is suitable for use as a
\glref{generalized~reference} that contains a \glref{proper~list} 
\arg[key] A \glref{function} of one argument, or \nil{} (default is \nil)
\arg[test] A \glref{function} of two arguments that returns a
\glref{generalized~boolean} (default is \nobr{\code{\#'eql}}) 
\arg[test-not] A \glref{function} of two arguments that returns a
\glref{generalized~boolean} (use of \nobr{\code{:test-not}} is deprecated)
\arg[new-place-value] A proper list
\end{args}

\fnreturns The new value of \var{place}. 

\begin{alsos}{as-atomic-operation}
\also[as-atomic-operation]
\also[atomic-delete]
\also[atomic-flush]
\also[atomic-pop]
\also[atomic-push]
\end{alsos}

\fnexamples
%
\W\supp
\begin{example}
  > list
  (1 2 3)
  > (atomic-pushnew 2 list)
  (1 2 3)
  > (atomic-pushnew 10 list)
  (10 1 2 3)
  >
\end{example}

\end{functiondoc}

%% ------------------------------------------------------------------------

\begin{functiondoc}{Function}{awaken-thread}{\var{thread\/}}
\index{thread!awakening}%
\index{awakening a thread}%

\fnsyntax

\fnpurpose Awaken a hibernating \glref{thread}.

\fnpackage \code{:portable-threads}

\fnmodule \code{:portable-threads}

\fnargs
\begin{args}{thread}
\arg[thread] A \glref{thread}
\end{args}

\fnerrors
\nothreads{}

\fndescription An attempt to awaken a non-hibernating thread is ignored.

\begin{alsos}{hibernate-thread}
\also[hibernate-thread]
\end{alsos}

\fnexample
%
\W\supp
\begin{example}
  (awaken-thread thread)
\end{example}

\end{functiondoc}

%% ------------------------------------------------------------------------

\begin{functiondoc}{Class}{condition-variable}{}
\index{class!condition-variable@\textbf{condition-variable}}%
  
\fnsyntax

\fnpackage \code{:portable-threads}

\fnmodule \code{:portable-threads}

\fndescription The class \nobr{\textbf{condition-variable}} is a subclass of
\nobr{\code{standard-object}}.  Instances of
\nobr{\textbf{condition-variable}} include an associated lock, which can be
either a \glref{lock} (the default) or a \glref{recursive~lock}.

\begin{alsos}{make-condition-variable}
\also[make-condition-variable]
\also[define-class]
\end{alsos}

\end{functiondoc}

%% ------------------------------------------------------------------------

\begin{functiondoc}{Function}{condition-variable-broadcast}%
  {\var{condition-variable}}
\index{signaling, condition variable!all blocked threads}%

\fnsyntax

\fnpurpose Unblock all \glref{threads} that are blocked on \nobr{\var{condition-variable}}.

\fnpackage \code{:portable-threads}

\fnmodule \code{:portable-threads}

\fnargs
\begin{args}{condition-variable}
\arg[condition-variable] A \glref{condition~variable}
\end{args}

\fnerrors
\nocvlock{}

\fndescription If no threads are blocked on \nobr{\var{condition-variable}},
this function is a no-op.

\begin{alsos}{condition-variable-wait-with-timeout}
\also[condition-variable-signal]
\also[condition-variable-wait]
\also[condition-variable-wait-with-timeout]
\also[make-condition-variable]
\also[with-lock-held]
\also[without-lock-held]
\end{alsos}

\bfindexit{with-lock-held}%
\fnexample Acquire the \glref{lock} associated with
\nobr{\code{condition-variable}} and then signal all blocked \glref{threads}
that are waiting on it:
%
\W\supp
\begin{example}
  (\entlink{with-lock-held} (condition-variable)
    (condition-variable-broadcast condition-variable))
\end{example}

\fnnote On Common Lisp implementations without threads, this function does
nothing.

\end{functiondoc}

%% ------------------------------------------------------------------------

\begin{functiondoc}{Function}{condition-variable-signal}%
  {\var{condition-variable}}
\index{signaling, condition variable!one blocked thread}%

\fnsyntax

\fnpurpose Unblock one \glref{thread} that is blocked on \nobr{\var{condition-variable}}.

\fnpackage \code{:portable-threads}

\fnmodule \code{:portable-threads}

\fnargs
\begin{args}{condition-variable}
\arg[condition-variable] A \glref{condition~variable}
\end{args}

\fnerrors
\nocvlock{}

\fndescription If no threads are blocked on \nobr{\var{condition-variable}},
this function is a no-op.

\begin{alsos}{condition-variable-wait-with-timeout}
\also[condition-variable-broadcast]
\also[condition-variable-wait]
\also[condition-variable-wait-with-timeout]
\also[make-condition-variable]
\also[with-lock-held]
\also[without-lock-held]
\end{alsos}

\bfindexit{with-lock-held}%
\fnexample Acquire the \glref{lock} associated with
\nobr{\code{condition-variable}} and then signal one blocked \glref{thread}
that is waiting on it:
%
\W\supp
\begin{example}
  (\entlink{with-lock-held} (condition-variable)
    (condition-variable-signal condition-variable))
\end{example}

\fnnote On Common Lisp implementations without threads, this function does
nothing.

\end{functiondoc}

%% ------------------------------------------------------------------------

\begin{functiondoc}{Function}{condition-variable-wait}%
  {\var{condition-variable}}
\index{waiting, on condition variable}%

\fnsyntax

\fnpurpose Block the current \glref{thread} on \nobr{\var{condition-variable}}.

\fnpackage \code{:portable-threads}

\fnmodule \code{:portable-threads}

\fnargs
\begin{args}{condition-variable}
\arg[condition-variable] A \glref{condition~variable}
\end{args}

\fnerrors
\nocvlock{}
\par
\nothreads{}

\begin{alsos}{condition-variable-wait-with-timeout}
\also[condition-variable-broadcast]
\also[condition-variable-signal]
\also[condition-variable-wait-with-timeout]
\also[make-condition-variable]
\also[with-lock-held]
\also[without-lock-held]
\end{alsos}

\bfindexit{with-lock-held}%
\fnexample Acquire the condition-variable \glref{lock} and then wait until
signaled by another \glref{thread}:
%
\W\supp
\begin{example}
  (\entlink{with-lock-held} (condition-variable)
    (condition-variable-wait condition-variable))
\end{example}

\end{functiondoc}

%% ------------------------------------------------------------------------

\begin{functiondoc}{Function}{condition-variable-wait-with-timeout}%
  {\var{condition-variable seconds}
    \returns{} \var{boolean\/}}

\index{waiting, on condition variable, time limited}%

\fnsyntax

\fnpurpose Block the current \glref{thread} on
\nobr{\var{condition-variable\/}} or until \nobr{\var{seconds\/}} seconds have
elapsed.

\fnpackage \code{:portable-threads}

\fnmodule \code{:portable-threads}

\fnargs
\begin{args}{condition-variable}
\arg[condition-variable] A \glref{condition~variable}
\arg[seconds] A number
\arg[boolean] A \glref{generalized~boolean}
\end{args}

\fnreturns True if \nobr{\var{condition-variable\/}} is unblocked before
\nobr{\var{seconds\/}} seconds have elapsed; \nil{} if the timeout has occurred.

\fnerrors
\nocvlock{}
\par
\nothreads{}

\begin{alsos}{condition-variable-broadcast}
\also[condition-variable-broadcast]
\also[condition-variable-signal]
\also[condition-variable-wait]
\also[make-condition-variable]
\also[with-lock-held]
\also[without-lock-held]
\end{alsos}

\bfindexit{with-lock-held}%
\fnexample Acquire the condition-variable \glref{lock} and then wait until
signaled by another \glref{thread} or until 5 seconds have elapsed:
%
\W\supp
\begin{example}
  (\entlink{with-lock-held} (condition-variable)
    (condition-variable-wait-with-timeout condition-variable 5))
\end{example}

\end{functiondoc}

%% ------------------------------------------------------------------------

\begin{functiondoc}{Function}{current-thread}{\noargs{} 
    \returns{} \var{thread\/}}
\index{thread!obtaining the current}%

\fnsyntax

\fnpurpose Return the object representing the current \glref{thread}.

\fnpackage \code{:portable-threads}

\fnmodule \code{:portable-threads}

\fnargs
\begin{args}{thread}
\arg[thread] A \glref{thread}
\end{args}

\fnreturns The object representing the current \glref{thread}. 

\begin{alsos}{spawn-thread}
\also[all-threads]
\also[spawn-form]
\also[spawn-thread]
\end{alsos}

\fnexample
%
\W\supp
\begin{example}
  > (current-thread)
  #<thread Listener 1>
  >
\end{example}

\fnnote On Common Lisp implementations without threads,
the keyword symbol \nobr{\code{:threads-not-available}} is returned.

\end{functiondoc}

%% ------------------------------------------------------------------------

\begin{functiondoc}{Function}{hibernate-thread}{\noargs{}}
\index{thread!hibernating}%

\fnsyntax

\fnpurpose Hibernate the current \glref{thread}.

\fnpackage \code{:portable-threads}

\fnmodule \code{:portable-threads}

\fnerrors
\nothreads{}

\begin{alsos}{awaken-thread}
\also[awaken-thread]
\end{alsos}

\fnexample
Hibernate the current \glref{thread}:
%
\W\supp
\begin{example}
  (hibernate-thread)
\end{example}

\end{functiondoc}

%% ------------------------------------------------------------------------

\begin{functiondoc}{Function}{kill-periodic-function}{\noargs{}}
\index{periodic function!terminating}%
\index{function!periodic, terminating}%

\fnsyntax

\fnpurpose Terminate the thread invoking a \glref{periodic~function}.

\fnpackage \code{:portable-threads}

\fnmodule \code{:portable-threads}

\fnerrors
\nothreads{}

\textbf{Kill-periodic-function} called outside the dynamic scope of a
\glref{periodic~function}.

\begin{alsos}{*periodic-function-verbose*}
\also[*periodic-function-verbose*]
\also[all-threads]
\also[kill-thread]
\also[spawn-periodic-function]
\end{alsos}

\bfindexit{define-event-class}%
\bfindexit{signal-event}%
\bfindexit{control-shell-running-p}%
\bfindexit{spawn-periodic-function}%
%
\fnexample Define and spawn a \glref{periodic~function} that is invoked every
0.5 seconds to signal a \nobr{\code{half-second-interrupt-event}}, continuing
as long as the control shell is running:
%
\W\supp
\begin{example}
  > (\entlink{define-event-class} half-second-timer-event (timer-interrupt-event)
      ())
  half-second-timer-event
  > (defun half-second-timer ()
      (unless (\entlink{control-shell-running-p})
        (kill-periodic-function))
      (\entlink{signal-event} 'half-second-timer-event))
  half-second-timer
  > (\entlink{spawn-periodic-function} 'half-second-timer 0.5)
  #<thread Periodic Function half-second-timer>
  >
\end{example}

\end{functiondoc}

%% ------------------------------------------------------------------------

\begin{functiondoc}{Function}{kill-thread}{\var{thread\/}}
\index{thread!killing}%
\index{killing a thread}%

\fnsyntax

\fnpurpose Kill a \glref{thread}.

\fnpackage \code{:portable-threads}

\fnmodule \code{:portable-threads}

\fnargs
\begin{args}{thread}
\arg[thread] A \glref{thread}
\end{args}

\fnerrors
\nothreads{}

\begin{alsos}{thread-alive-p}
\also[spawn-form]
\also[spawn-thread]
\also[thread-alive-p]
\end{alsos}

\fnexample
%
\W\supp
\begin{example}
  (kill-thread thread)
\end{example}

\end{functiondoc}

%% ------------------------------------------------------------------------

\begin{functiondoc}{Function}{make-condition-variable}%
  {\code{\&rest} \var{initargs\/} \\ 
   \code{\&key} \var{class lock\/} \\
   \returns{} \var{condition-variable\/}}
\index{creating!a condition variable}%
\index{making!a condition variable}%
\index{condition variable!creating}%
 
\fnsyntax

\fnpurpose Create a new \glref{condition~variable}.

\fnpackage \code{:portable-threads}

\fnmodule \code{:portable-threads}

\fnargs
\begin{args}{condition-variable}
\arg[initargs] An \glref{initialization~argument~list}
\arg[class] The name of the \glref{class} for the created
\glref{condition-variable} instance (default is
\nobr{\code{condition-variable}})
\arg[lock] A \glref{lock}, a \glref{recursive~lock}, or a 
\glref{condition~variable} (default is a non-recursive lock)
\arg[condition-variable] A \glref{condition~variable}
\end{args}

\fnreturns
The created \nobr{\textbf{\entlink{condition-variable}}}.

\begin{alsos}{make-recursive-lock}
\also[make-instance]
\also[make-lock]
\also[make-recursive-lock]
\also[with-lock-held]
\also[without-lock-held]
\end{alsos}

\fnexamples
Make a \nobr{\textbf{condition-variable}} instance with a non-recursive 
\glref{lock}:
%
\W\supp
\begin{example}
  > (make-condition-variable)
  #<condition-variable>
  >
\end{example}
%
Make a \nobr{\textbf{condition-variable}} instance with a
\glref{recursive~lock}:
%
\W\supp\notpretop
\begin{example}
  > (make-condition-variable :lock (\entlink{make-recursive-lock}))
  #<condition-variable>
  >
\end{example}
%
Define a subclass of \nobr{\textbf{condition-variable}} that includes a 
\nobr{\code{state}} slot:
%
\W\supp\notpretop
\begin{example}
  (defclass state-cv (condition-variable)
    ((state :initarg :state
            :initform nil
            :accessor state-of)))
\end{example}
%
and then create a \nobr{\code{state-cv}} instance with a
\glref{recursive~lock}:
%
\W\supp\notpretop
\begin{example}
  > (make-condition-variable :class 'state-cv
                             :lock (\entlink{make-recursive-lock}))
  #<state-cv>
  >
\end{example}

\fnnote The \textbf{make-condition-variable} function is equivalent to using
\nobr{\textbf{\entlinknoex{make-instance}}} with the desired \glref{class} for
the created \glref{condition-variable} instance.  However, using
\nobr{\textbf{make-condition-variable}} is preferable stylistically.

\end{functiondoc}

%% ------------------------------------------------------------------------

\begin{functiondoc}{Function}{make-lock}{\code{\&key} \var{name\/}
    \returns{} \var{lock\/}} 
\index{lock!creating}%
\index{creating!a lock}%
\index{making!a lock}%

\fnsyntax

\fnpurpose Create a \glref{lock}.

\fnpackage \code{:portable-threads}

\fnmodule \code{:portable-threads}

\fnargs
\begin{args}{name}
\arg[name] A string.
\arg[lock] A \glref{lock}
\end{args}

\fnreturns The newly created \glref{lock}. 

\begin{alsos}{make-condition-variable}
\also[make-condition-variable]
\also[make-recursive-lock]
\also[thread-holds-lock-p]
\also[with-lock-held]
\also[without-lock-held]
\end{alsos}

\fnexample
%
\W\supp
\begin{example}
  > (make-lock :name "Priority Queue")
  #<lock Priority Queue>
  >
\end{example}

\fnnote On Common Lisp implementations without threads, a
``pseudo-lock'' object is returned.

\end{functiondoc}

%% ------------------------------------------------------------------------

\begin{functiondoc}{Function}{make-recursive-lock}{\code{\&key} \var{name\/}
    \returns{} \var{recursive-lock\/}} 
\index{lock!creating}%
\index{creating!a lock}%
\index{making!a lock}%

\fnsyntax

\fnpurpose Create a \glref{recursive~lock}.

\fnpackage \code{:portable-threads}

\fnmodule \code{:portable-threads}

\fnargs
\begin{args}{name}
\arg[name] A string.
\arg[lock] A \glref{recursive~lock}
\end{args}

\fnreturns The newly created \glref{recursive~lock}. 

\begin{alsos}{make-condition-variable}
\also[make-condition-variable]
\also[make-lock]
\also[thread-holds-lock-p]
\also[with-lock-held]
\also[without-lock-held]
\end{alsos}

\fnexample
%
\W\supp
\begin{example}
  > (make-recursive-lock :name "Priority Queue")
  #<recursive-lock Priority Queue>
  >
\end{example}

\fnnote On Common Lisp implementations without threads, a
``pseudo-recursive-lock'' object is returned.

\end{functiondoc}

%% ------------------------------------------------------------------------

\begin{functiondoc}{Function}{make-scheduled-function}{\var{function\/} 
    \code{\&key} \var{name key test\/}
    \returns{} \var{scheduled-function\/}} 
\index{scheduled function!creating}%
\index{function!scheduled, creating}%
\index{creating!a scheduled function}%
\index{making!a scheduled function}%

\fnsyntax

\fnpurpose Create a \glref{scheduled~function}.

\fnpackage \code{:portable-threads}

\fnmodule \code{:portable-threads}

\fnargs
\begin{args}{scheduled-function}
\arg[function] A \glref{function~designator} specifying a
  \glref{function~object} of one argument 
\arg[name] An object (typically a
  string or a symbol; default is \nobr{\var{function}}, if
  \nobr{\var{function\/}} is a symbol, otherwise \nil)
\arg[key] An object (default is \nil)
\arg[test] A \glref{function} of two arguments that returns a
\glref{generalized~boolean} (default is \nobr{\code{\#'eql}})
\arg[scheduled-function] A \glref{scheduled~function}
\end{args}

\fnreturns The newly created \glref{scheduled~function}. 

\fnerrors
\nothreads{}

\fndescription Unless the run time to perform \nobr{\var{function\/}} is
brief, it should spawn a new \glref{thread} in which to perform its activities
so as to avoid delaying the invocation of a subsequent
\glref{scheduled~function}.

The optional \var{key\/} and associated key-comparison \var{test\/} can be
specified to distinguish \glref{scheduled~functions} with the same \var{name}
or \var{function}.

\begin{alsos}{*schedule-function-verbose*}
\also[*schedule-function-verbose*]
\also[all-scheduled-functions]
\also[schedule-function]
\also[schedule-function-relative]
\also[scheduled-function-key]
\also[scheduled-function-name]
\also[scheduled-function-repeat-interval]
\also[scheduled-function-test]
\also[spawn-periodic-function]
\also[unschedule-function]
\end{alsos}

\fnexamples
%
Create a \glref{scheduled~function} that simply prints \nobr{\code{"Hello"}}
when invoked:
%
\W\supp
\begin{example}
  > (make-scheduled-function 
      #'(lambda (scheduled-function)
          (declare (ignore scheduled-function))
          (print "Hello"))
      :name 'hello)
  #<scheduled-function hello [unscheduled]>
  >
\end{example}

\bfindexit{scheduled-function-repeat-interval}%
\bfindexit{spawn-thread}%
%
A more complex \glref{scheduled~function} that spawns a new \glref{thread} to
do its work and randomly sets whether to reschedule itself (and at what
interval):
%
\W\supp
\begin{example}
  > (defun complex-function (scheduled-function)
      (let ((interval (random 100)))
        (setf (\entlink{scheduled-function-repeat-interval} scheduled-function)
              (if (plusp interval) 
                  ;; repeat 1-99 seconds from now:
                  interval
                  ;; don't repeat 1\% of the time:
                  nil)))
      (\entlink{spawn-thread} "Lots of stuff doer" #'do-lots-of-stuff))
  complex-function
  > (make-scheduled-function 'complex-function)
  #<scheduled-function complex-function [unscheduled]>
  >
\end{example}

\end{functiondoc}

%% ------------------------------------------------------------------------

\begin{functiondoc}[nearly-forever-seconds]{Constant}%
  {nearly-forever-seconds}{}%

\codeindexit{sleep}%

\bfindexit{nearly-forever-seconds}%

\fnsyntax

\fnpurpose The maximum number of seconds supported by \code{sleep}.

\fnpackage \code{:portable-threads}

\fnmodule \code{:portable-threads}

\fnvaluetype A \glref{fixnum}

\fnvalue Implementation dependent

\begin{alsos}{sleep-nearly-forever}
\also[sleep-nearly-forever]
\end{alsos}

\end{functiondoc}

%% ------------------------------------------------------------------------

\begin{functiondoc}{Function}{restart-scheduled-function-scheduler}%
  {\noargs{} \returns{} \var{thread\/}}
\index{scheduled function!scheduler, restarting}%

\fnsyntax

\fnpurpose Restart the \glref{scheduled-function} scheduling thread.

\fnpackage \code{:portable-threads}

\fnmodule \code{:portable-threads}

\fnargs
\begin{args}{thread}
\arg[thread] A \glref{thread} or \nil{}
\end{args}

\fnreturns The object representing the newly spawned
\glref{scheduled-function} scheduler \glref{thread} or \nil{} if the
\glref{scheduled-function} scheduler was already running.

\fnerrors
\nothreads{}

\fndescription If the \glref{scheduled-function} scheduler \glref{thread} has
been killed accidentally, this function can be used to start a new scheduler
\glref{thread}.

\begin{alsos}{scheduled-function-repeat-interval}
\also[schedule-function]
\also[scheduled-function-repeat-interval]
\also[unschedule-function]
\end{alsos}

\fnexamples
Restart the \glref{scheduled-function} scheduler:
%
\W\supp
\begin{example}
  > (restart-scheduled-function-scheduler)
  #<thread Scheduled-Function Scheduler>
  >
\end{example}
%
Restarting a \glref{scheduled-function} scheduler that is already running has
no effect:
%
\W\supp\notpretop
\begin{example}
  > (restart-scheduled-function-scheduler)
  ;; The scheduled-function scheduler is already running.
  nil
  >
\end{example}

\end{functiondoc}

%% ------------------------------------------------------------------------

\begin{functiondoc}{Function}{run-in-thread}{\var{thread function\/}
    \code{\&rest} \var{args\/}}
\index{thread!running a function in}%

\fnsyntax

\fnpurpose Force \var{thread\/} to apply \nobr{\var{function\/}} to \nobr{\var{args\/}}

\fnpackage \code{:portable-threads}

\fnmodule \code{:portable-threads}

\fnargs
\begin{args}{function}
\arg[thread] A \glref{thread}
\arg[function] A \glref{function~designator}
\arg[args] Arguments to the function
\end{args}

\fnerrors
\nothreads{}

\begin{alsos}{spawn-thread}
\also[spawn-form]
\also[spawn-thread]
\end{alsos}

\fnexample
%
\W\supp
\begin{example}
  (run-in-thread thread
                 #'(lambda (result) (throw ':exit result)) 
                 result)
\end{example}

\end{functiondoc}

%% ------------------------------------------------------------------------

\begin{functiondoc}{Function}{schedule-function}%
  {\var{name-or-scheduled-function invocation-time\/} 
    \code{\&key} \var{key repeat-interval verbose\/}}
\index{scheduled function!scheduling}%
\index{function!scheduled, scheduling}%
\index{scheduling!a scheduled function}%

\fnsyntax

\fnpurpose Schedule a \glref{scheduled~function} at an absolute invocation time.

\fnpackage \code{:portable-threads}

\fnmodule \code{:portable-threads}

\fnargs
\begin{args}{name-or-scheduled-function}
\arg[name-or-scheduled-function] An object (typically a string or a
  symbol) naming a currently scheduled \glref{scheduled~function} or a
  \nobr{\code{scheduled-function}} object
\arg[invocation-time] A \glref{universal~time}
\arg[key] An object (default is \nil)
\arg[repeat-interval] A positive integer (representing seconds) or
  \nil{} (default is \nil)
\arg[verbose] A \glref{generalized~boolean} 
  (default is \nobr{\textbf{\entlink{*schedule-function-verbose*}}})
\end{args}

\fnerrors
\nothreads{}

\fndescription If the \nobr{\var{scheduled-function\/}} object is unscheduled,
it is added to the list of currently scheduled \glref{scheduled~functions}
with the specified \nobr{\var{invocation-time\/}} and
\nobr{\var{repeat-interval}}.  If the \nobr{\var{scheduled-function\/}} object
is currently scheduled, it is first unscheduled and then rescheduled with the
specified \nobr{\var{invocation-time\/}} and \nobr{\var{repeat-interval}}.

\begin{alsos}{restart-scheduled-function-scheduler}
\also[*schedule-function-verbose*]
\also[all-scheduled-functions]
\also[encode-time-of-day]
\also[make-scheduled-function]
\also[restart-scheduled-function-scheduler]
\also[schedule-function-relative]
\also[scheduled-function-repeat-interval]
\also[spawn-periodic-function]
\also[unschedule-function]
\end{alsos}

\bfindexit{make-scheduled-function}%
\fnexamples Schedule a \glref{scheduled~function} that simply prints
\nobr{\code{"Happy New Year!"}}  at midnight (local time) on January 1, 2010:
%
\W\supp
\begin{example}
  > (schedule-function
      (\entlink{make-scheduled-function}
        #'(lambda (scheduled-function)
            (declare (ignore scheduled-function))
            (print "Happy New Year!")))
       (encode-universal-time 0 0 0 1 1 2010))
  > (all-scheduled-functions)
  (#<scheduled-function nil [Jan 1, 2010 00:00:00]>)
  >
\end{example}
%
\bfindexit{make-scheduled-function}%
\bfindexit{encode-time-of-day}%
%
Schedule a \glref{scheduled~function} that prints \nobr{\code{"It's quitting
    time!"}}  every day at 5pm:
%
\W\supp\notpretop
\begin{example}
  > (schedule-function
      (\entlink{make-scheduled-function}
        #'(lambda (scheduled-function)
            (declare (ignore scheduled-function))
            (print "It's quitting time!"))
        :name 'quitting-time)
       (\entlink{encode-time-of-day} 17 0 0) :repeat-interval #.(* 24 60 60))
  >
\end{example}
%
\bfindexit{encode-time-of-day}%
%
Verbosely change \nobr{\code{quitting-time}} to 5:30pm every day:
%
\W\supp\notpretop
\begin{example}
  > (schedule-function 'quitting-time (\entlink{encode-time-of-day} 17 30 0)
      :repeat-interval #.(* 24 60 60)
      :verbose 't)
  ;; Unscheduling #<scheduled-function quitting-time [17:00:00]>...
  ;; Scheduling #<scheduled-function quitting-time [17:30:00]> 
  ;; as the next scheduled-function...
  >
\end{example}

\end{functiondoc}

%% ------------------------------------------------------------------------

\begin{functiondoc}{Function}{schedule-function-relative}%
  {\var{name-or-scheduled-function seconds\/} 
    \code{\&key} \var{key repeat-interval verbose\/}}
\index{scheduled function!scheduling}%
\index{function!scheduled, scheduling}%
\index{scheduling!a scheduled function}%

\fnsyntax

\fnpurpose Schedule a \glref{scheduled~function} a specified number of
seconds from now.

\fnpackage \code{:portable-threads}

\fnmodule \code{:portable-threads}

\fnargs
\begin{args}{name-or-scheduled-function}
\arg[name-or-scheduled-function] An object (typically a string or a
  symbol) naming a currently scheduled \glref{scheduled~function} or a
  \nobr{\code{scheduled-function}} object
\arg[seconds] A positive integer
\arg[key] An object (default is \nil)
\arg[repeat-interval] A positive integer (representing seconds) or
  \nil{} (default is \nil)
\arg[verbose] A \glref{generalized~boolean}
  (default is \nobr{\textbf{\entlink{*schedule-function-verbose*}}})
\end{args}

\fnerrors
\nothreads{}

\fndescription If the \nobr{\var{scheduled-function\/}} object is unscheduled,
it is added to the list of currently scheduled \glref{scheduled~functions}
with an invocation time of \nobr{\var{interval\/}} seconds from the current
time and the specified \nobr{\var{repeat-interval}}.  If the
\nobr{\var{scheduled-function\/}} object is currently scheduled, it is first
unscheduled and then rescheduled with an invocation time of
\nobr{\var{interval\/}} seconds from the current time and the specified
\nobr{\var{repeat-interval}}.

\begin{alsos}{restart-scheduled-function-scheduler}
\also[*schedule-function-verbose*]
\also[all-scheduled-functions]
\also[make-scheduled-function]
\also[restart-scheduled-function-scheduler]
\also[schedule-function]
\also[scheduled-function-invocation-time]
\also[scheduled-function-key]
\also[scheduled-function-name]
\also[scheduled-function-repeat-interval]
\also[scheduled-function-test]
\also[spawn-periodic-function]
\also[unschedule-function]
\end{alsos}

\bfindexit{make-scheduled-function}%
\fnexamples
%
Schedule a \glref{scheduled~function} that simply prints \nobr{\code{"Hello!"}} 
5 seconds from now:
%
\W\supp
\begin{example}
  > (schedule-function-relative
      (\entlink{make-scheduled-function}
        #'(lambda (scheduled-function)
            (declare (ignore scheduled-function))
            (print "Hello!")))
       5)
  >
\end{example}
%
\bfindexit{make-scheduled-function}%
\bfindexit{signal-event}%
%
Schedule a \glref{scheduled~function} that signals a GBBopen
\nobr{\code{timer-interrupt-event}} every 30 seconds:
%
\W\supp\notpretop
\begin{example}
  > (schedule-function-relative
      (\entlink{make-scheduled-function}
        #'(lambda (scheduled-function)
            (declare (ignore scheduled-function))
            (\entlink{signal-event} 'timer-interrupt-event)))
      30
      :repeat-interval 30)
  >
\end{example}

\fnnote The form \nobr{\code{(schedule-function-relative}
  \nobr{\var{scheduled-function\/}} \code{10)}} is equivalent to
\nobr{\code{(\entlink{schedule-function}} \var{scheduled-function\/}
\code{(+ (get-universal-time) 10))}}.

\end{functiondoc}

%% ------------------------------------------------------------------------

\begin{functiondoc}{Function}{scheduled-function-invocation-time}%
  {\var{scheduled-function\/}
    \returns{} \var{invocation-time\/}}

\index{function!scheduled, invocation time}%

\fnsyntax

\fnpurpose Return the invocation time of a \glref{scheduled~function}.

\fnpackage \code{:portable-threads}

\fnmodule \code{:portable-threads}

\fnargs
\begin{args}{scheduled-function}
\arg[scheduled-function] A \glref{scheduled~function}
\arg[invocation-time] A \glref{universal~time}
\end{args}

\fnreturns The invocation time of \var{scheduled-function}. 

\begin{alsos}{schedule-function-relative}
\also[all-scheduled-functions]
\also[make-scheduled-function]
\also[schedule-function]
\also[schedule-function-relative]
\also[scheduled-function-key]
\also[scheduled-function-name]
\also[scheduled-function-repeat-interval]
\also[scheduled-function-test]
\end{alsos}

\fnexample
%
Return the invocation-time of \glref{scheduled~function}
\nobr{\code{scheduled-function}}:
%
\W\supp
\begin{example}
  > (scheduled-function-invocation-time scheduled-function)
  3465679813
  >
\end{example}

\end{functiondoc}

%% ------------------------------------------------------------------------

\begin{functiondoc}{Function}{scheduled-function-key}%
  {\var{scheduled-function\/}
    \returns{} \var{key\/}}
\index{function!scheduled, key}%

\fnsyntax

\fnpurpose Return the key of a \glref{scheduled~function}.

\fnpackage \code{:portable-threads}

\fnmodule \code{:portable-threads}

\fnargs
\begin{args}{scheduled-function}
\arg[scheduled-function] A \glref{scheduled~function}
\arg[key] An object
\end{args}

\fnreturns The key of \var{scheduled-function}. 

\begin{alsos}{schedule-function-relative}
\also[all-scheduled-functions]
\also[make-scheduled-function]
\also[schedule-function]
\also[schedule-function-relative]
\also[scheduled-function-name]
\also[scheduled-function-repeat-interval]
\also[scheduled-function-test]
\end{alsos}

\bfindexit{all-scheduled-functions}%
\fnexample
%
Return the key of all currently scheduled \glref{scheduled~functions}:
%
\W\supp
\begin{example}
  > (mapcar #'scheduled-function-key (\entlink{all-scheduled-functions}))
  (nil)
  >
\end{example}

\end{functiondoc}

%% ------------------------------------------------------------------------

\begin{functiondoc}{Function}{scheduled-function-name}%
  {\var{scheduled-function\/}
    \returns{} \var{name\/}}
\index{function!scheduled, name}%

\fnsyntax

\fnpurpose Return the name of a \glref{scheduled~function}.

\fnpackage \code{:portable-threads}

\fnmodule \code{:portable-threads}

\fnargs
\begin{args}{scheduled-function}
\arg[scheduled-function] A \glref{scheduled~function}
\arg[name] An object (typically a string or a symbol)
\end{args}

\fnreturns The name of \var{scheduled-function}. 

\begin{alsos}{schedule-function-relative}
\also[all-scheduled-functions]
\also[make-scheduled-function]
\also[schedule-function]
\also[schedule-function-relative]
\also[scheduled-function-key]
\also[scheduled-function-repeat-interval]
\also[scheduled-function-test]
\end{alsos}

\bfindexit{all-scheduled-functions}%
\fnexample
%
Return the names of all currently scheduled \glref{scheduled~functions}:
%
\W\supp
\begin{example}
  > (mapcar #'scheduled-function-name (\entlink{all-scheduled-functions}))
  (quitting-time)
  >
\end{example}

\end{functiondoc}

%% ------------------------------------------------------------------------

\begin{functiondoc}{Function}{scheduled-function-repeat-interval}%
  {\var{scheduled-function\/}
    \returns{} \var{repeat-interval\/}}
\index{function!scheduled, repeat-interval value}%
\index{changing!repeat-interval, of a scheduled function}%

\fnsyntax

\fnpurpose Return the repeat interval of a \glref{scheduled~function}.

\fnsetf
\fnsetfsyntax{scheduled-function-repeat-interval}%
  {\var{scheduled-function\/}}{\var{repeat-interval\/}}

\fnpackage \code{:portable-threads}

\fnmodule \code{:portable-threads}

\fnargs
\begin{args}{scheduled-function}
\arg[scheduled-function] A \glref{scheduled~function}
\arg[repeat-interval] A positive integer (representing seconds) or \nil
\end{args}

\fnreturns The repeat interval of \var{scheduled-function}. 

\begin{alsos}{schedule-function-relative}
\also[all-scheduled-functions]
\also[make-scheduled-function]
\also[schedule-function]
\also[schedule-function-relative]
\also[scheduled-function-key]
\also[scheduled-function-name]
\also[scheduled-function-test]
\end{alsos}

\bfindexit{all-scheduled-functions}%
\fnexamples
% 
Display the \nobr{\code{scheduled-function}} object and its repeat interval
for each currently scheduled \glref{scheduled~function}:
%
\W\supp
\begin{example}
  > (dolist (scheduled-function (\entlink{all-scheduled-functions}))
     (format t "~&;; ~s ~s~%"
             scheduled-function
             (scheduled-function-repeat-interval scheduled-function)))
  ;; #<scheduled-function quitting-time [17:00:00]> 86400
  nil
  >
\end{example}

Define a \glref{function} to be used as a \glref{scheduled~function} that
randomly sets whether to reschedule itself (and at what interval):
%
\W\supp
\begin{example}
  (defun complex-function (scheduled-function)
    (let ((interval (random 100)))
      (setf (scheduled-function-repeat-interval scheduled-function)
            (if (plusp interval) 
                ;; repeat 1-99 seconds from now:
                interval
                ;; don't repeat 1\% of the time:
                nil)))
    (do-some-stuff))
\end{example}

\end{functiondoc}

%% ------------------------------------------------------------------------

\begin{functiondoc}{Function}{scheduled-function-test}%
  {\var{scheduled-function\/}
    \returns{} \var{key\/}}
\index{function!scheduled, test}%

\fnsyntax

\fnpurpose Return the key-comparison test of a \glref{scheduled~function}.

\fnpackage \code{:portable-threads}

\fnmodule \code{:portable-threads}

\fnargs
\begin{args}{scheduled-function}
\arg[scheduled-function] A \glref{scheduled~function}
\arg[test] A \glref{function} of two arguments that returns a
\glref{generalized~boolean}
\end{args}

\fnreturns The key-comparison test of \var{scheduled-function}. 

\begin{alsos}{schedule-function-relative}
\also[all-scheduled-functions]
\also[make-scheduled-function]
\also[schedule-function]
\also[schedule-function-relative]
\also[scheduled-function-key]
\also[scheduled-function-invocation-time]
\also[scheduled-function-name]
\end{alsos}

\bfindexit{all-scheduled-functions}%
\fnexample
%
Return the key-comparison tests of all currently scheduled
\glref{scheduled~functions}:
%
\W\supp
\begin{example}
  > (mapcar #'scheduled-function-key (\entlink{all-scheduled-functions}))
  (#'eql)
  >
\end{example}

\end{functiondoc}

%% ------------------------------------------------------------------------

\begin{functiondoc}{Function}{sleep-nearly-forever}{\code{\&optional}
    \var{seconds\/}}

\fnsyntax

\fnpurpose A maximum-time-bounded \code{sleep}.

\fnpackage \code{:portable-threads}

\fnmodule \code{:portable-threads}

\fnargs
\begin{args}{function}
\arg[seconds] An integer (default is 
  \nobr{\textbf{\entlink{nearly-forever-seconds}}})
\end{args}

\fndescription 
%
\codeindexit{sleep}%
\bfindexit{nearly-forever-seconds}%
%
Calls \code{sleep} with \var{seconds\/} or \nobr{\textbf{\entlink{nearly-forever-seconds}}}, whichever is less.  Using \nobr{\textbf{nearly-forever-seconds}} protects against exceeding the duration limit of the Common Lisp implementation's \code{sleep} function for very long duration sleeping, by truncating the duration to \nobr{\textbf{\entlink{nearly-forever-seconds}}}.


\begin{alsos}{nearly-forever-seconds}
\also[nearly-forever-seconds]
\end{alsos}

\fnexamples
%
\W\supp
\begin{example}
  > (sleep-nearly-forever)    ; sleep for a very long time
        ...
  > (sleep-nearly-forever     ; sleep as long as the above
       (* 2 most-positive-fixnum))
        ...
  >
\end{example}

\end{functiondoc}

%% ------------------------------------------------------------------------

\begin{functiondoc}{Macro}{spawn-form}{\var{form\/}\superstar
  \returns{} \var{thread\/}}
\index{thread!spawning}%
\index{creating!a thread}%
\index{making!a thread}%

\fnsyntax

\fnpurpose Evaluate \var{forms\/} in a new \glref{thread}.

\fnpackage \code{:portable-threads}

\fnmodule \code{:portable-threads}

\fnargs
\begin{args}{function}
\arg[form] A \glref{form}
\arg[thread] A \glref{thread}
\end{args}

\fnreturns The object representing the new \glref{thread}.
  
\fnerrors
\nothreads{}

\begin{alsos}{symbol-value-in-thread}
\also[all-threads]
\also[awaken-thread]
\also[current-thread]
\also[hibernate-thread]
\also[kill-thread]
\also[spawn-thread]
\also[thread-alive-p]
\also[thread-name]
\also[thread-whostate]
\also[threadp]
\also[run-in-thread]
\also[symbol-value-in-thread]
\end{alsos}

\fnexample
%
\W\supp
\begin{example}
  > (spawn-form (sleep 60))
  #<thread Form (sleep 60)>
  >
\end{example}

\end{functiondoc}

%% ------------------------------------------------------------------------

\begin{functiondoc}{Function}{spawn-periodic-function}%
  {\var{function repeat-interval\/} 
    \code{\&key} \var{count name verbose\/}
    \returns{} \var{thread\/}}
\index{periodic function!spawning}%
\index{function!periodic, spawning}%

\fnsyntax

\fnpurpose Spawn a thread invoking \nobr{\var{function\/}} every
\nobr{\var{repeat-interval\/}} seconds.

\fnpackage \code{:portable-threads}

\fnmodule \code{:portable-threads}

\fnargs
\begin{args}{repeat-interval}
\arg[function] A \glref{function~designator} specifying a
  \glref{function~object} of no arguments
\arg[repeat-interval] A number (representing seconds)
\arg[count] A number or \nil{} (default is \nil)
\arg[name] An object (typically a string or a symbol; default is
\nobr{\var{function}}, if \nobr{\var{function\/}} is a symbol, otherwise \nil)
\arg[verbose] A \glref{generalized~boolean}
  (default is \nobr{\textbf{\entlink{*periodic-function-verbose*}}})
\arg[thread] A \glref{thread}
\end{args}

\fnreturns The object representing the \glref{thread} associated with the
\glref{periodic~function}.

\fnerrors
\nothreads{}

\fndescription If \var{count\/} is \nil, \nobr{\var{function\/}} will continue
to be invoked every \nobr{\var{repeat-interval\/}} seconds until the
\glref{periodic-function} \glref{thread} is killed or until
\nobr{\var{function\/}} calls
\nobr{\textbf{\entlink{kill-periodic-function}}}.  Otherwise,
\nobr{\var{count\/}} is decremented by one prior to each invocation of
\nobr{\var{function\/}} and, if it is negative, the \glref{periodic~function}
is terminated.

\begin{alsos}{*periodic-function-verbose*}
\also[*periodic-function-verbose*]
\also[all-threads]
\also[kill-periodic-function]
\also[kill-thread]
\also[make-scheduled-function]
\also[schedule-function]
\also[schedule-function-relative]
\end{alsos}

\fnexamples 
Spawn a simple \glref{periodic~function} that is invoked every 0.1 seconds,
but that only runs twice:
%
\W\supp
\begin{example}
  > (spawn-periodic-function #'(lambda () (print "Hello!")) 0.1 
      :name 'hello
      :count 2)
  #<thread Periodic Function hello>
  >
  "Hello!" 
  "Hello!" 
\end{example}
%
\bfindexit{kill-periodic-function}%
%
Spawn a simple \glref{periodic~function} that is invoked every 0.1 seconds
that runs up to 20 times, but with a 10\% chance on each invocation of
terminating early:
%
\W\supp\notpretop
\begin{example}
  > (spawn-periodic-function 
       #'(lambda ()
           (when (zerop (random 10))
             (kill-periodic-function))
           (print "Hello!")) 
       0.1
       :count 20
       :verbose 't)
  ;; Spawning periodic-function thread for...
  #<thread Periodic Function>
  >
  "Hello!" 
  "Hello!" 
  "Hello!" 
  "Hello!" 
  ;; Killing periodic-function...
  ;; Exiting periodic-function thread
\end{example}
%
\bfindexit{define-event-class}%
\bfindexit{signal-event}%
\bfindexit{control-shell-running-p}%
\bfindexit{kill-periodic-function}%
%
Define and spawn a \glref{periodic~function} that is invoked every 0.5 seconds
to signal a \nobr{\code{half-second-interrupt-event}}, continuing as long as
the control shell is running:
%
\W\supp\notpretop
\begin{example}
  > (\entlink{define-event-class} half-second-timer-event (timer-interrupt-event)
      ())
  half-second-timer-event
  > (defun half-second-timer ()
      (unless (\entlink{control-shell-running-p})
        (\entlink{kill-periodic-function}))
      (\entlink{signal-event} 'half-second-timer-event))
  half-second-timer
  > (spawn-periodic-function 'half-second-timer 0.5)
  #<thread Periodic Function half-second-timer>
  >
\end{example}

\end{functiondoc}

%% ------------------------------------------------------------------------

\begin{functiondoc}{Function}{spawn-thread}{\var{name function\/}
    \code{\&rest} \var{args\/} 
    \returns{} \var{thread\/}}
\index{thread!spawning}%
\index{creating!a thread}%
\index{making!a thread}%

\fnsyntax

\fnpurpose Spawn a new \glref{thread}.

\fnpackage \code{:portable-threads}

\fnmodule \code{:portable-threads}

\fnargs
\begin{args}{function}
\arg[name] A string
\arg[function] A \glref{function~designator}
\arg[args] Arguments to the function
\arg[thread] A \glref{thread}
\end{args}

\fnreturns The object representing the new \glref{thread}.
  
\fnerrors
\nothreads{}

\begin{alsos}{symbol-value-in-thread}
\also[all-threads]
\also[awaken-thread]
\also[current-thread]
\also[hibernate-thread]
\also[kill-thread]
\also[spawn-form]
\also[thread-alive-p]
\also[thread-name]
\also[thread-whostate]
\also[threadp]
\also[run-in-thread]
\also[symbol-value-in-thread]
\end{alsos}

\fnexample
%
\W\supp
\begin{example}
  > (spawn-thread "Sleepy" #'sleep 60)
  #<thread Sleepy>
  >
\end{example}

\end{functiondoc}

%% ------------------------------------------------------------------------

\begin{functiondoc}{Function}{symbol-value-in-thread}{\var{symbol 
    thread\/}
    \returns{} \nobr{\var{object, boundp\/}}}
\index{thread!symbol value in}%
\index{value, of a symbol in a thread}%

\fnsyntax

\fnpurpose Return the value of \var{symbol\/} in a \glref{thread}.

\fnpackage \code{:portable-threads}

\fnmodule \code{:portable-threads}

\fnargs
\begin{args}{function}
\arg[symbol] A symbol
\arg[thread] A \glref{thread}
\arg[object] An object
\arg[boundp] A \glref{generalized~boolean}
\end{args}

\fnreturns Two values:
\begin{tightitemize}
\item the value of \var{symbol\/} in \nobr{\var{thread\/}} or nil if no value
  is bound
\item \code{t} if \var{symbol\/} is specially or globally bound in 
  \nobr{\var{thread}}; otherwise \nil
\end{tightitemize}
  
\fndescription The global symbol value is returned as the first value if no
thread-local value is bound.

\begin{alsos}{spawn-thread}
\also[spawn-form]
\also[spawn-thread]
\end{alsos}

\fnexamples
%
\W\supp
\begin{example}
  > (symbol-value-in-thread '*x* thread)
  33
  t
  > (symbol-value-in-thread 'pi thread)
  3.141592653589793d0
  t
  > (symbol-value-in-thread '*unbound* thread)
  nil
  nil
  >
\end{example}

\fnnote On Common Lisp implementations without threads, this function obtains
the global symbol value.

\end{functiondoc}

%% ------------------------------------------------------------------------

\begin{functiondoc}{Function}{thread-alive-p}{\var{thread\/} 
    \returns{} \var{boolean\/}}

\fnsyntax

\fnpurpose Return a value indicating whether a \glref{thread} is alive.

\fnpackage \code{:portable-threads}

\fnmodule \code{:portable-threads}

\fnargs
\begin{args}{thread}
\arg[thread] A \glref{thread}
\arg[boolean] A \glref{generalized~boolean}
\end{args}

\fnreturns True if \var{thread\/} is alive; \nil{} otherwise.

\fnerrors
\nothreads{}

\begin{alsos}{spawn-thread}
\also[all-threads]
\also[kill-thread]
\also[spawn-form]
\also[spawn-thread]
\end{alsos}

\bfindexit{spawn-thread}%
\bfindexit{kill-thread}%
\fnexamples
%
\W\supp
\begin{example}
  > (defparameter *silly-thread* (\entlink{spawn-thread} "Sleeper" 'sleep 10000))
  #<thread Sleeper>
  > (thread-alive-p *silly-thread*)
  t
  > (\entlink{kill-thread} *silly-thread*)
  t
  > (thread-alive-p *silly-thread*)
  nil
  >
\end{example}

\end{functiondoc}

%% ------------------------------------------------------------------------

\begin{functiondoc}{Function}{thread-name}{\var{thread\/} 
   \returns{} \var{name-string\/}} 

\fnsyntax

\fnpurpose Return the name of a \glref{thread}.

\fnsetf
\fnsetfsyntax{thread-name}{\var{thread\/}}{\var{name-string\/}}

\fnpackage \code{:portable-threads}

\fnmodule \code{:portable-threads}

\fnargs
\begin{args}{thread}
\arg[thread] A \glref{thread}
\arg[name-string] A string
\end{args}

\fnreturns The name of \var{thread}.

\fnerrors
\nothreads{}

\begin{alsos}{spawn-thread}
\also[spawn-form]
\also[spawn-thread]
\end{alsos}

\fnexamples
%
\W\supp
\begin{example}
  > (thread-name thread)
  "Initial"
  > (setf (thread-name thread) "Version 2")
  "Version 2"
  > (thread-name thread)
  "Version 2"
  >
\end{example}

\fnnote Digitool's \xsitelink{Macintosh Common Lisp}{http://www.digitool.com}
does not support changing the thread name via \nobr{\textbf{setf}}.

\end{functiondoc}

%% ------------------------------------------------------------------------

\begin{functiondoc}{Function}{thread-whostate}{\var{thread\/}
   \returns{} \var{whostate\/}} 

\fnsyntax

\fnpurpose Return a string that describes the current state of a 
\glref{thread}.

\fnsetf
\fnsetfsyntax{thread-whostate}%
  {\var{thread\/}}{\var{whostate\/}}

\fnpackage \code{:portable-threads}

\fnmodule \code{:portable-threads}

\fnargs
\begin{args}{whostate}
\arg[thread] A \glref{thread}
\arg[whostate] A string or \nil{}
\end{args}

\fnreturns The whostate string of the \glref{thread} or \nil.

\fnerrors
\nothreads{}

\begin{alsos}{spawn-thread}
\also[spawn-form]
\also[spawn-thread]
\end{alsos}

\fnexample
%
\W\supp
\begin{example}
  > (thread-whostate thread)
  "Running"
  >
\end{example}

\fnnote Although the \var{whostate\/} value can provide helpful information
when debugging, specific \nobr{\var{whostate\/}} values and their meanings
vary among Common Lisp implementations and should not be used
programmatically.

Only \xsitelink{Allegro CL}{http://www.franz.com}, \xsitelink{Clozure
  CL}{http://clozure.com/clozurecl.html}, and Digitool's \xsitelink{Macintosh
  Common Lisp}{http://www.digitool.com} support user-settable whostates;
\nobr{\textbf{(setf~whostate)}} is a no-op on other Common Lisp
implementations.

\end{functiondoc}

%% ------------------------------------------------------------------------

\begin{functiondoc}{Function}{thread-yield}{\noargs{}}
\index{thread!yielding to other threads}%
\index{yielding to other threads}%

\fnsyntax

\fnpurpose Give other \glref{threads} a chance to execute.

\fnpackage \code{:portable-threads}

\fnmodule \code{:portable-threads}

\fnexample
%
\W\supp
\begin{example}
  (thread-yield)
\end{example}

\fnnote On Common Lisp implementations without thread support, this function
executes \nobr{\textbf{\entlink{run-polling-functions}}} if the
\nobr{\code{:polling-functions}} module has been loaded.  Otherwise, it is a
no-op on non-threaded implementations.

\end{functiondoc}

%% ------------------------------------------------------------------------

\begin{functiondoc}{Function}{threadp}{\var{object\/} 
    \returns{} \var{boolean\/}}
\index{thread!checking state}%

\fnsyntax

\fnpurpose Check if \var{object\/} is an object representing a
\glref{thread}.

\fnpackage \code{:portable-threads}

\fnmodule \code{:portable-threads}

\fnargs
\begin{args}{object}
\arg[object] An object
\arg[boolean] A \glref{generalized~boolean}
\end{args}

\fnreturns True if \var{object\/} is an object representing a 
\glref{thread}; \nil{} otherwise.

\begin{alsos}{thread-alive-p}
\also[all-threads]
\also[spawn-form]
\also[spawn-thread]
\also[thread-alive-p]
\end{alsos}

\bfindexit{all-threads}%
\fnexample
%
\W\supp
\begin{example}
  > (threadp (car (\entlink{all-threads})))
  t
  >
\end{example}

\end{functiondoc}

%% ------------------------------------------------------------------------

\begin{functiondoc}{Function}{thread-holds-lock-p}{\var{lock\/}
    \returns{} \var{boolean\/}}
\index{lock, held by current thread}%

\fnsyntax

\fnpurpose Determine if \var{lock\/} is held by the current \glref{thread}.

\fnpackage \code{:portable-threads}

\fnmodule \code{:portable-threads}

\fnargs
\begin{args}{boolean}
\arg[lock] A \glref{lock}, a \glref{recursive~lock}, or a 
\glref{condition~variable}
\arg[boolean] A \glref{generalized~boolean}
\end{args}

\fnreturns True if the current thread holds \var{lock}; \nil{} otherwise.

\begin{alsos}{make-condition-variable}
\also[make-condition-variable]
\also[make-lock]
\also[make-recursive-lock]
\also[with-lock-held]
\also[without-lock-held]
\end{alsos}

\bfindexit{with-lock-held}%
\fnexamples
Two simple examples using a \glref{lock}:
%
\W\supp
\begin{example}
  > (thread-holds-lock-p lock)
  nil
  > (\entlink{with-lock-held} (lock)
      (thread-holds-lock-p lock))
  t
  >
\end{example}
%
Two more simple examples using a \glref{condition~variable}:
%
\W\supp\notpretop
\begin{example}
  > (thread-holds-lock-p condition-variable)
  nil
  > (\entlink{with-lock-held} (condition-variable)
      (thread-holds-lock-p condition-variable))
  t
  >
\end{example}

\end{functiondoc}

%% ------------------------------------------------------------------------

\begin{functiondoc}{Function}{unschedule-function}%
  {\var{name-or-scheduled-function\/}
    \code{\&key} \var{key verbose\/}
  \returns{} \var{boolean\/}} 
\index{scheduled function!canceling scheduling}%
\index{function!scheduled, canceling scheduling}%
\index{scheduling!canceling a scheduled function}%
\index{unscheduling!a scheduled function}%

\fnsyntax

\fnpurpose Cancel the upcoming invocation of a currently scheduled
\glref{scheduled~function}.

\fnpackage \code{:portable-threads}

\fnmodule \code{:portable-threads}

\fnargs
\begin{args}{name-or-scheduled-function}
\arg[name-or-scheduled-function] An object (typically a string or a
  symbol) naming a currently scheduled \glref{scheduled~function} or a
  \nobr{\code{scheduled-function}} object
\arg[key] An object (default is \nil)
\arg[verbose] A \glref{generalized~boolean} 
  (default is \nobr{\textbf{\entlink{*schedule-function-verbose*}}})
\arg[boolean] A \glref{generalized~boolean}
\end{args}

\fnreturns The \glref{scheduled~function} if it was unscheduled; \nil{} if the 
\glref{scheduled~function} was not currently scheduled or was not found.

\fnerrors
\nothreads{}

\fndescription If the \nobr{\var{scheduled-function\/}} object is scheduled,
it is removed from the list of currently scheduled
\glref{scheduled~functions}.

\begin{alsos}{restart-scheduled-function-scheduler}
\also[*schedule-function-verbose*]
\also[all-scheduled-functions]
\also[make-scheduled-function]
\also[schedule-function]
\also[schedule-function-relative]
\end{alsos}

\fnexamples
Unschedule the \nobr{\code{quitting-time}} \glref{scheduled~function}:
%
\W\supp
\begin{example}
  > (unschedule-function 'quitting-time)
  #<scheduled-function quitting-time [unscheduled]>
  >
\end{example}
%
\bfindexit{all-scheduled-functions}%
%
Unschedule all currently scheduled \glref{scheduled~functions}:
%
\W\supp\notpretop
\begin{example}
  > (\entlink{all-scheduled-functions})
  (#<scheduled-function nil [Jan 1, 2010 00:00:00]>)
  > (mapc #'unschedule-function (\entlink{all-scheduled-functions}))
  (#<scheduled-function nil [unscheduled]>)
  > (\entlink{all-scheduled-functions})
  nil
  >
\end{example}
%
Unschedule a non-existent \glref{scheduled~function}:
%
\W\supp\notpretop
\begin{example}
  > (unschedule-function 'non-existent)
  ;; Warning: Scheduled-function non-existent was not scheduled; no action taken.
  nil
  >
\end{example}

\end{functiondoc}

%% ------------------------------------------------------------------------

\begin{functiondoc}{Macro}{with-lock-held}{\code{(}\var{lock\/} 
    \code{\&key}
    \var{whostate\/}\code{)}
    \var{form\/}\superstar{} 
    \returns{} \var{result\/}\superstar}
\index{lock!acquiring}%
\index{acquiring!a lock}%
\index{recursive lock!acquiring}%
\index{acquiring!a recursive lock}%

\fnsyntax

\fnpurpose After acquiring a \glref{lock} or a \glref{recursive~lock},
execute forms and then release the lock.

\fnpackage \code{:portable-threads}

\fnmodule \code{:portable-threads}

\fnargs
\begin{args}{whostate}
\arg[lock] A \glref{lock}, a \glref{recursive~lock}, or a 
\glref{condition~variable}
\arg[whostate] A string (default \nobr{\code{"With Lock Held"}})
\arg[forms] An implicit \nobr{\textbf{progn}} of \glref{forms} to be evaluated
\arg[results] The values returned by evaluating the last \var{form}
\end{args}

\fnreturns The values returned by evaluating the last \var{form}.

\fnerrors A \glref{thread} attempts to re-acquire a (non-recursive)
\glref{lock} that it holds.
  
\fndescription If a \glref{thread} executes a \nobr{\textbf{with-lock-held}}
that is dynamically inside another \nobr{\textbf{with-lock-held}} involving
the same \glref{recursive~lock}, the inner \nobr{\textbf{with-lock-held}}
simply proceeds as if it had acquired the lock.

\begin{alsos}{thread-holds-lock-p}
\also[make-condition-variable]
\also[make-lock]
\also[make-recursive-lock]
\also[thread-holds-lock-p]
\also[thread-whostate]
\also[without-lock-held]
\end{alsos}

\fnexamples
Acquire the lock controlling access to a critical section of code:
%
\W\supp
\begin{example}
  (with-lock-held (lock :whostate "Waiting for Critical Lock")
    (critical-section))
\end{example}
%
A silly example showing a recursive re-acquisition of a
\glref{recursive~lock}:
%
\W\supp\notpretop
\begin{example}
  (with-lock-held (recursive-lock :whostate "Waiting for Critical Lock")
    (with-lock-held (recursive-lock :whostate "Again Waiting for Critical Lock")
      (critical-section)))
\end{example}
%
\bfindexit{condition-variable-signal}%
%
Acquire the \glref{lock} associated with \nobr{\code{condition-variable}} and
then signal all blocked \glref{threads} that are waiting on it:
%
\W\supp\notpretop
\begin{example}
  (with-lock-held (condition-variable)
    (\entlink{condition-variable-signal} condition-variable))
\end{example}

\fnnote The \var{whostate\/} value is ignored by
\xsitelink{SBCL}{http://sbcl.sourceforge.net}.

\end{functiondoc}

%% ------------------------------------------------------------------------

\begin{functiondoc}{Macro}{with-timeout}{\code{(}\var{seconds\/} 
    \var{timeout-form\/}\superstar{}\code{)}
    \var{form\/}\superstar{} 
    \returns{} \var{result\/}\superstar}
\fnsyntax

\fnpurpose Bound the time allowed to evaluate \nobr{\var{forms\/}} to
\nobr{\var{seconds}}, evaluating \nobr{\var{timeout-forms\/}} if the time
limit is reached.

\fnpackage \code{:portable-threads}

\fnmodule \code{:portable-threads}

\fnargs
\begin{args}{whostate}
\arg[seconds] A number
\arg[timeout-forms] An implicit \nobr{\textbf{progn}} of \glref{forms} to be
evaluated if the timed \nobr{\var{forms\/}} do not complete before \nobr{\var{seconds\/}} seconds have elapsed
\arg[forms] An implicit \nobr{\textbf{progn}} of \glref{forms} to be evaluated
\arg[results] The values returned by evaluating the last \var{form}
\end{args}

\fnreturns The values returned by evaluating the last \var{form\/} if
completed in less than \nobr{\var{seconds\/}} seconds; otherwise the values
returned by evaluating the last \nobr{\var{timeout-form\/}}

\fnerrors \nothreads{} However, \nobr{\textbf{with-timeout}} is also
supported on non-threaded \xsitelink{SBCL}{http://sbcl.sourceforge.net}.

\fndescription If the evaluation of \nobr{\var{forms\/}} does not complete
within \nobr{\var{seconds\/}} seconds, execution of \nobr{\var{forms\/}} is
terminated and the \nobr{\var{timeout-forms\/}} are evaluated, returning the
result of the last \nobr{\var{timeout-form}}. The \nobr{\var{timeout-forms\/}}
are not evaluated if the \nobr{\var{forms\/}} complete within
\nobr{\var{seconds\/}} seconds, in which case the result of the last
\var{form\/} is returned.

\begin{alsos}{condition-variable-wait-with-timeout}
\also[condition-variable-wait-with-timeout]
\end{alsos}

\fnexamples
Evaluate a simple form, with a one-second time out:
%
\W\supp
\begin{example}
  > (with-timeout (1 ':timed-out) 
       ':did-not-time-out)
  :did-not-time-out
  >
\end{example}
%
Again, but this time sleep for two seconds to cause a time out:
%
\W\supp\notpretop
\begin{example}
  > (with-timeout (1 ':timed-out)
       (sleep 2) 
       ':did-not-time-out)
  :timed-out              \textrm{\textcolor{green}{; (after 2 seconds)}}
  >
\end{example}

\end{functiondoc}

%% ------------------------------------------------------------------------

\begin{functiondoc}{Macro}{without-lock-held}{\code{(}\var{lock\/} 
    \code{\&key}
    \var{whostate\/}\code{)}
    \var{form\/}\superstar{} 
    \returns{} \var{result\/}\superstar}
\index{lock!releaseing temporarily}%
\index{releasing!a lock, temporarily}%
\index{recursive lock!releasing temporarily}%
\index{releasing!a recursive lock, temporarily}%

\fnsyntax

\fnpurpose Temporarily release a \glref{lock} or a \glref{recursive~lock},
execute forms and then reacquire the lock.

\fnpackage \code{:portable-threads}

\fnmodule \code{:portable-threads}

\fnargs
\begin{args}{whostate}
\arg[lock] A \glref{lock}, a \glref{recursive~lock}, or a 
\glref{condition~variable}
\arg[whostate] A string (default \nobr{\code{"Without Lock Held"}})
\arg[forms] An implicit \nobr{\textbf{progn}} of \glref{forms} to be evaluated
\arg[results] The values returned by evaluating the last \var{form}
\end{args}

\fnreturns The values returned by evaluating the last \var{form}.

\fnerrors A \glref{thread} attempts to release a \glref{lock} that it does not
hold.
  
\begin{alsos}{thread-holds-lock-p}
\also[make-condition-variable]
\also[make-lock]
\also[make-recursive-lock]
\also[thread-holds-lock-p]
\also[thread-whostate]
\also[with-lock-held]
\end{alsos}

\fnexample 
%
\bfindexit{with-lock-held}%
%
Acquire and temporarily release a lock controlling access to several
critical sections of code:
%
\W\supp
\begin{example}
  (\entlink{with-lock-held} (lock :whostate "Waiting for Critical Lock")
    (critical-section-1)
    (without-lock-held (lock :whostate "Doing non-critical stuff")
      (non-critical-section))
    (critical-section-2))
\end{example}

\fnnote The \var{whostate\/} value is ignored by
\xsitelink{SBCL}{http://sbcl.sourceforge.net}.

\end{functiondoc}

%% ========================================================================
