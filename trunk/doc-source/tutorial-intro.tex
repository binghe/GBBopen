%% -*- Mode:TeX; Fonts:(hl12fb) -*-
%% *-* File: /home/gbbopen/current/doc-source/tutorial-intro.tex *-*
%% *-* Last-Edit: Tue Dec 11 03:46:12 2007; Edited-By: cork *-*
%% *-* Machine: ruby.corkills.org *-*

%% Copyright (C) 2006-2006, Dan Corkill <corkill@GBBopen.org>
%% Part of the GBBopen Project (see LICENSE for license information).

%% ========================================================================
%%  Shared Acknowledgments text

GBBopen is a high-performance, open source blackboard-system framework.  This
tutorial shows you how to get started using GBBopen through a series of
exercises that cover GBBopen's concepts and features in a step-by-step
sequence.  The exercises guide you in creating a simple ``random walk''
application that simulates taking a sequence of straight-line excursions, each
of random length and direction.  Although the application is simple, it
involves many of GBBopen's features, from very basic to fairly advanced.

\T\subsection*{GBBopen and Common Lisp}
\W\textbf{\large GBBopen and Common Lisp}

GBBopen is an extension of \xsitelink{Common
  Lisp}{http://www.lispworks.com/documentation/HyperSpec/Front/index.htm} and
uses CLOS (the Common Lisp Object System) and the \xsitelink{Metaobject
  Protocol}{http://www.lisp.org/mop/index.html} (MOP) to provide
blackboard-specific object capabilities.  The blending of GBBopen with Common
Lisp transfers all the advantages of a rich, dynamic, reflective, and
extensible programming language to blackboard-application developers.  Thus,
GBBopen's ``programming language'' includes all of Common Lisp in addition to
the blackboard-system extensions documented in the \textit{GBBopen Reference}
manual.

This tutorial does not attempt to teach Common Lisp programming, and an
understanding of basic Common Lisp and CLOS concepts is assumed.  Although it
is possible to read through the tutorial exercises without Common Lisp
expertise, a much deeper understanding of GBBopen's potential is gained by
understanding how GBBopen and Common Lisp work smoothly together.  Two
frequently recommended Common Lisp books are Peter Seibel's
\xsitelink{\textit{Practical Common Lisp\/}}{http://www.gigamonkeys.com/book/}
and Paul Graham's \xsitelink{\textit{On
    Lisp\/}}{http://paulgraham.com/onlisptext.html}.  Both books are available
on line, as well in traditional book form.  A less programmer-oriented
introduction to Common Lisp is David Touretzky's \xsitelink{\textit{Common
    Lisp: A Gentle Introduction to Symbolic
    Computation\/}}{http://www.cs.cmu.edu/~dst/LispBook/}, also available on
line. 

Ken Pitman's \xsitelink{\textit{Common Lisp
    HyperSpec\/}}{http://www.lispworks.com/documentation/HyperSpec/}, an
easily navigable HTML document derived from the ANSI Common Lisp standard, is
the customary programmer's reference for Common Lisp.  A
\xsitelink{downloadable 
archive}{ftp://ftp.lispworks.com/pub/software_tools/reference/HyperSpec-7-0.tar.gz}
of the HyperSpec is also available, which is very convenient when working
without a continuous connection to the Internet.  We will show how to make the
HyperSpec and the GBBopen Reference HyperDoc directly accessible in your
Common Lisp development environment in
\texorhtml{Exercise~\ref{sec:environment}.}{the \link{Enhancing Your
Development Environment}{sec:environment} exercise.}

\T\subsection*{Using a Common Lisp/GBBopen development environment}
\W\textbf{\large Using a Common Lisp/GBBopen development environment}

The tutorial exercises build on one another, and they are intended to be
performed sequentially.  The initial exercise involves installing GBBopen and
preparing it for use in your environment.  This is followed by several
exercises where you interact with GBBopen by entering forms into the ``Lisp
Listener'' (also called the read-eval-print-loop or simply the REPL) that is
provided by your Common Lisp implementation.  As the scope of the random-walk
application grows, however, it is important to set up a working environment
where your work is done in a file.  So, after these initial GBBopen exercises,
we will spend an exercise setting up your environment to provide you
power-user productivity for the remainder of the tutorial (and for future
GBBopen activities).  This diversion exercise will be worth your time!

%% ========================================================================
