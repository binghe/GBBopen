%% -*- Mode:TeX; Fonts:(hl12fb) -*-
%% *-* File: /usr/local/gbbopen/doc-source/common.tex *-*
%% *-* Last-Edit: Tue May 27 16:08:20 2008; Edited-By: cork *-*
%% *-* Machine: cyclone.cs.umass.edu *-*

%% Copyright (C) 2003-2008, Dan Corkill <corkill@GBBopen.org>
%% Part of the GBBopen Project (see LICENSE for license information).
%%
%% ========================================================================
%%  The LaTeX and hyperlatex code used in producing GBBopen documentation
%%  is placed under and covered by  the GBBopen software license that 
%%  accompanies each GBBopen distribution and is also available at
%%  http://GBBopen.org/downloads/LICENSE. 
%% ========================================================================

%% Timezone & Daylight Savings:
%\newcommand{\timezone}{EST}
\newcommand{\timezone}{EDT}

%% Html declarations: output directory and filenames, node title
\newcommand{\gbbopenversion}{1.0}
\newcommand{\runningtitle}{GBBopen \gbbopenversion{} \docname}
\htmltitle{\runningtitle}

\newcommand{\noargs}{$<$no arguments$>$}
\T\newcommand{\bkslash}{$\backslash$}
\W\newcommand{\bkslash}{\backslash}
\T\newcommand{\vbar}{$|$}
\W\newcommand{\vbar}{|}
\T\newcommand{\superstar}{\raisebox{2pt}{*}}
\W\newcommand{\superstar}{*}
\T\newcommand{\superplus}{\raisebox{4pt}{+}}
\W\newcommand{\superplus}{\xml{sup}+\xml{/sup}}
\definecolor{rulegray}{gray}{.75}
\definecolor{darkergray}{rgb}{.4,.5,.4}
\definecolor{darkgreen}{rgb}{.0,.7,.0}
\definecolor{verydarkgreen}{rgb}{.0,.5,.0}
\definecolor{caution}{rgb}{.8,.6,.0}

%% ----------------------------------------------------------------------------
%%   Margins

\T\headsep 26pt
\T\footskip 36pt
\T\topmargin -40pt
\T\oddsidemargin 0pt
\T\evensidemargin 0pt
\T\textwidth 470pt
\T\textheight 630pt

%% ----------------------------------------------------------------------------
%%   Headings/Footings

\W\begin{iftex}
\fancyhf{} % clear all
\fancyhead[er,ol]{\textcolor{darkergray}{\rightmark}}
\fancyfoot[er,ol]{\textcolor{darkergray}{\runningtitle}
                \\\textcolor{darkergray}{\nouppercase{\leftmark}}}
\fancyfoot[el,or]{~\\\thepage}

\makeatletter

%% from fancyhdr
\def\ps@fancy{%
\@ifundefined{@chapapp}{\let\@chapapp\chaptername}{}%for amsbook
%
% Define \MakeUppercase for old LaTeXen.
% Note: we used \def rather than \let, so that \let\uppercase\relax (from
% the version 1 documentation) will still work.
%
\@ifundefined{MakeUppercase}{\def\MakeUppercase{\uppercase}}{}%
\@ifundefined{chapter}{\def\sectionmark##1{\markboth
{\ifnum \c@secnumdepth>\z@
 \thesection\hskip 1em\relax \fi ##1}{}}%
\def\subsectionmark##1{\markboth {\ifnum \c@secnumdepth >\@ne
 \thesubsection\hskip 1em\relax \fi ##1}{}}}%
{\def\chaptermark##1{\markboth {\MakeUppercase{\ifnum \c@secnumdepth>\m@ne
 \@chapapp\ \thechapter. \ \fi ##1}}{}}%
\def\sectionmark##1{\markboth{\ifnum \c@secnumdepth >\z@
 \thesection. \ \fi ##1}{}}}%
\ps@@fancy
\gdef\ps@fancy{\@fancyplainfalse\ps@@fancy}%
\ifdim\headwidth<0sp
    \global\advance\headwidth123456789sp\global\advance\headwidth\textwidth
\fi}
\def\ps@fancyplain{\ps@fancy \let\ps@plain\ps@plain@fancy}
\def\ps@plain@fancy{\@fancyplaintrue\ps@@fancy}
\let\ps@@empty\ps@empty

%% mod of \headrule from fancyhdr.sty (to make rule gray):
\def\headrule{{\if@fancyplain\let\headrulewidth\plainheadrulewidth\fi
    \vskip-0.8\baselineskip % tighten rule to \rightmark
    \textcolor{rulegray}{\hrule\@height\headrulewidth\@width\headwidth 
    \vskip-\headrulewidth}}}

%% mod of \ps@@fancy from fancyhdr.sty:
\def\ps@fancybottom{%
\def\@mkboth{\protect\markboth}%
\def\@oddhead{}%
\def\@oddfoot{\@fancyfoot\fancy@Oolf\f@ncyolf\f@ncyocf\f@ncyorf\fancy@Oorf}%
\def\@evenhead{}%
\def\@evenfoot{\@fancyfoot\fancy@Oelf\f@ncyelf\f@ncyecf\f@ncyerf\fancy@Oerf}%
}
\makeatother
\W\end{iftex}

\T\begin{ifhtml}
\newcommand\entities{\textbf{\large Entities}}
\T\end{ifhtml}

%% ----------------------------------------------------------------------------
%%  LaTeX Modifications

\W\begin{iftex}
\newcommand{\goodpagebreak}{\pagebreak[0]}
\makeatletter

% article.cls
\renewcommand\@idxitem{\par\hangindent 30\p@}
\renewcommand\subitem{\@idxitem \hspace*{10\p@}}
\renewcommand\subsubitem{\@idxitem \hspace*{20\p@}}
% ppar
\newcommand{\ppar}[1]{\paragraph{#1}~\\[0.5\baselineskip]%
  \T\addcontentsline{toc}{subsubsection}{{#1}}
}
% Tighten up example environment (uses atbeginend.sty):
\AfterBegin{example}{\vspace*{-0.8\baselineskip}}

% fndocsec
\newcommand{\fndocsec}{\@startsection{subparagraph}{5}{\z@}%
                          {1.25ex \@plus1ex \@minus.1ex}%
                          {-1em}%
                          {\normalfont\large\sffamily\bfseries}}
% from article.cls - Increase TOC font for sections and subsections
\renewcommand*\l@section[2]{%
  \ifnum \c@tocdepth >\z@
    \addpenalty\@secpenalty
    \addvspace{1.0em \@plus\p@}%
    \setlength\@tempdima{2.0em}%
    \begingroup
      \parindent \z@ \rightskip \@pnumwidth
      \parfillskip -\@pnumwidth
      \leavevmode \bfseries
      \advance\leftskip\@tempdima
      \hskip -\leftskip
      {\large #1}\nobreak\hfil \nobreak\hb@xt@\@pnumwidth{\hss\large #2}\par
    \endgroup
    \addvspace{.5em \@plus\p@}%
  \fi}
\renewcommand*\l@subsection[2]{%
  \ifnum \c@tocdepth >\z@
    \addpenalty\@secpenalty
    \addvspace{0.4em \@plus\p@}%
    \setlength\@tempdima{2.3em}%
    \begingroup
      \parindent \z@ \rightskip \@pnumwidth
      \parfillskip -\@pnumwidth
      \leavevmode \bfseries
      \advance\leftskip\@tempdima
      \hskip -\leftskip
      #1\nobreak\hfil \nobreak\hb@xt@\@pnumwidth{\hss #2}\par
    \endgroup
    \addvspace{.2em \@plus\p@}%
  \fi}

% from article.cls - Decrease TOC leftmargin for subsubsections
\renewcommand*\l@subsubsection{\@dottedtocline{3}{3.3em}{2.3em}}

% from index.sty - Remove section* and more
\renewenvironment{theindex}{%
% \edef\indexname{\the\@nameuse{idxtitle@\@indextype}}%
  \if@twocolumn
  \@restonecolfalse
  \else
  \@restonecoltrue
  \fi
  \columnseprule \z@
  \columnsep 35\p@
% \twocolumn[%
% \section*{\indexname}%
% \ifx\index@prologue\@empty\else
%     \index@prologue
%     \bigskip
% \fi
% ]%
  \@mkboth{\MakeUppercase\indexname}%
          {\MakeUppercase\indexname}%
% \thispagestyle{plain}%
  \parindent\z@
  \parskip\z@ \@plus .3\p@\relax
  \let\item\@idxitem
}{%
  \if@restonecol
    \onecolumn
  \else
    \clearpage
  \fi
}

\makeatother
\W\end{iftex}

%% ----------------------------------------------------------------------------
%%  HyperLaTeX Modifications

\T\begin{ifhtml}

  %% No easy CSS hook for this (we don't want to mess up the navbar
  %% alignment):
  \xmlattributes{td}{valign="baseline"}

  \newcommand{\textrm}[1]{\xml{span 
    style="font-family: serif; font-style: normal"}{#1}\xml{/span}}
  \newcommand{\textsf}[1]{\xml{span 
    style="font-family: sans-serif; font-style: normal"}{#1}\xml{/span}}

  %% This should be the URL for the icons used in the navigation panels
  %% must end with a slash, unless you leave it empty
  %% (empty means the icons are in the same directory as the HTML file)
  \renewcommand{\HlxIcons}{}
  \renewcommand{\HlxStyleSheet}{}

  \renewcommand{\HlxMeta}{\xml{link rel="SHORTCUT ICON" href="favicon.ico"}}

  %% Switch `Next' and `Previous' on Panel
  \renewcommand{\HlxTopPanel}{\EmptyP{\HlxSeqPrevUrl\HlxUpUrl\HlxSeqNextUrl}{%
    \xml*{table width="100%" cellpadding="0" cellspacing="2"}\xml{tr}
    \xml*{td bgcolor="##99ccff"}%
    \EmptyP{\HlxSeqPrevUrl}{\xlink{\HlxImage{\HlxSeqPrevTitle}{previous.png}}%
      {\HlxSeqPrevUrl}}%
    {\htmlimg{\HlxIcons{}blank.png}{}}%
    \xml*{/td}%
    \xml*{td bgcolor="##99ccff"}%
    \EmptyP{\HlxUpUrl}
    {\xlink{\HlxImage{\HlxUpTitle}{up.png}}%
      {\HlxUpUrl}}%
    {\htmlimg{\HlxIcons{}blank.png}{}}%
    \xml*{/td}%
    \xml*{td bgcolor="##99ccff"}%
    \EmptyP{\HlxSeqNextUrl}{\xlink{\HlxImage{\HlxSeqNextTitle}{next.png}}%
      {\HlxSeqNextUrl}}%
    {\htmlimg{\HlxIcons{}blank.png}{}}%
    \xml*{/td}%
    \xml*{td align="center" bgcolor="##99ccff" width="100%"}%
    \textbf{\HlxThisTitle}%
    \xml*{/td}%
    \HlxPanelFields
    \xml*{/tr}%
    \xml*{/table}}{}}

  %% Add `GoTo Top' to Panel
  \renewcommand{\HlxPanelFields}{\xml*{td bgcolor="##99ccff"}%
    \GoToTopTarget%
    \HlxImage{GoTo Top}{top.png}%
    \xml{/a}%
    \xml*{/td}}

  \newcommand{\HlxAddressDatetime}{}

  %% Rename frames as "left" and "right"):
  \renewcommand{\HlxFramesDescription}[2]{
  \xml{frameset rows="100%" cols="25%,75%"}
    \xml{frame src="#1_toc#2" name="left" marginwidth="5"
      marginheight="5" scrolling="auto" border="0"}
    \xml{frame src="#1_0#2" name="right" marginwidth="20" 
      marginheight="20"}
    \xml{noframes}
    This document uses frames to assist navigation.
    Your browser is currently not supporting the use of frames, but you 
    may still access the 
    \xml{a target="_top" href="#1_0#2"}non-framed version\xml{/a}.
    \xml{/noframes}
    \xml{/frameset}}
  
  %% Add logo to Navigation Panel (and rename frame as "left"):
  \renewcommand{\HlxFramesNavigation}{%
    \HlxTocName
    \htmlpanel{0}%
    \HlxSection{-5}{}*{\xml{img align="left" src="GBBopen-logo-sm.png"}%
      \xml{br clear="both"}%
      {\large \navigationname}}%
    % \xml{base target="right"}% IE requires base to be in <HEAD>
    \htmlmenu[0]{1}
    \renewcommand{\HlxAddressDatetime}{,\xml{nobr}~\today{}\xml{/nobr}
      \xml{nobr}\hhmm~\timezone\xml{/nobr}}
    \renewcommand{\bottommatter}{}}

\renewcommand{\HlxBottomMatter}{%
  \HlxBlk\htmlrule\EmptyP{\HlxAddress}%
  {\xml{address}\HlxAddress%
  \EmptyP{\HlxAddressDatetime}{\HlxAddressDatetime}{}%
  \HlxBlk\xml{/address}\\}{}}

\T\end{ifhtml}

% Things to skip/fake in HyperLaTeX
\W\newcommand{\vspace}[1]{}
\W\newcommand{\bigskip}{\\}
\W\newcommand{\medskip}{\\}
\W\newcommand{\smallskip}{\\}
\W\newcommand{\vfill}{\\}
\W\newcommand{\hfill}{\xml{p align="right"}}
\W\newcommand{\normalbaselines}{}
\W\newcommand{\^}{\xmlent{##94}}
\W\newcommand{\infty}{\xmlent{##8734}}
\W\newcommand{\goodpagebreak}{}
%\W\NotSpecial{\do\^}

%% ----------------------------------------------------------------------------
%%  Special link & index commands
%%
%%  Note that \entlink and index commands can insert a space in hyperlatex
%%  format

% The following doesn't work in LaTeX example environments, due to 
% kerning/hyphenation:
%\T\newcommand{\entlink}[1]{\hyperref[ent:#1]{#1}}
% This doesn't get there either, but shows the problem with already
% kerned hyphen in the example environment:
%\T\newcommand{\entlink}[1]{\bgroup\let\-\@@hyph\typeout{ent:#1}#1\egroup}
\newcommand{\entlink}[1]{\link{#1}{ent:#1}}
% Special non-example \entlink for LaTeX:
\T\newcommand{\entlinknoex}[1]{\hyperref[ent:#1]{#1}}
\W\newcommand{\entlinknoex}[1]{\entlink{#1}}
%
\newcommand{\indexit}[1]{\index{#1|itidx}}
\newcommand{\bfindex}[1]{\index{#1@\textbf{#1}}}
\newcommand{\bfindexit}[1]{\index{#1@\textbf{#1}|itidx}}
\T\newcommand{\bfindexstart}[1]{\index{#1@\textbf{#1}|(}} %match |
\T\newcommand{\bfindexend}[1]{\index{#1@\textbf{#1}|)}}
\newcommand{\codeindex}[1]{\cindex[#1]{\code{#1}}}
\newcommand{\codeindexqual}[2]{\index{#1, #2@\code{#1}, #2}}
\newcommand{\codeindexsub}[2]{\index{#1,#2@\code{#1}!#2}}
\newcommand{\codeindexit}[1]{\index{#1@\code{#1}|itidx}}
\newcommand{\codeindexqualit}[2]{\index{#1, #2@\code{#1}, #2|itidx}}
\newcommand{\codeindexsubit}[2]{\index{#1,#2@\code{#1}!#2|itidx}}
\newcommand{\REPLindex}[1]{\index{#1@\textbf{#1} REPL command}%
                           \index{REPL command!#1@\textbf{#1}}}
\newcommand{\REPLindexit}[1]{\index{#1@\textbf{#1} REPL command|itidx}
                             \index{REPL command!#1@\textbf{#1}|itidx}}
\newcommand{\REPLindexnd}[2]{%
  \index{#1@\textbf{#1} REPL command (#2)|see{the GBBopen Tutorial}}%
  \index{REPL command!#1@\textbf{#1} (#2)|see{the GBBopen Tutorial}}}
\W\newcommand{\itidx}[2]{\xlink{\textcolor{verydarkgreen}{\textit{#1}}}{#2}}
\T\newcommand{\itidx}[1]{\textcolor{verydarkgreen}{\textit{#1}}}

%% Internal links (includes paper-format \ref references)
\W\newcommand{\reflink}[2]{\link{#1}{#2}}
\T\newcommand{\reflink}[2]{\link{#1}{#2} (see page~\pageref{#2})}
\W\newcommand{\refsectionlink}[2]{\link{#1}{#2}}
\T\newcommand{\refsectionlink}[2]{\link{#1}{#2} (Section~\ref{#2})}

%% X-links (tutorial to hyperdoc)
\T\begin{ifhtml}
  \newcommand{\xentlink}[2]{\xml{a target="_top"
                                   href="../hyperdoc/#2.html"}#1\xml{/a}}
\T\end{ifhtml}
\T\newcommand{\xentlink}[2]{} % working version needed

\newcommand{\xreflink}[2]{#1} % working versions needed
%\W\newcommand{\xreflink}[2]{} 
%\T\newcommand{\xreflink}[2]{}

%% Links to external sites
\W\newcommand{\xsitelink}[2]{\xml{a target="_top" href="#2"}#1\xml{/a}}
\T\newcommand{\xsitelink}[2]{\href{#2}{#1}}

%% ----------------------------------------------------------------------------
%%  Useful Lengths

\T\newlength{\tlengtha}  %% These are temporaries used in 
\T\newlength{\tlengthb}  %% functiondoc environment and
\T\newlength{\tlengthc}  %% elsewhere.

%% ----------------------------------------------------------------------------
%% ppar
\W\newcommand{\ppar}[1]{\endpar\xml{h4}#1\xml{/h4}}

%% ----------------------------------------------------------------------------

\newcommand{\entered}[1]{\textcolor{darkgreen}{[\textit{{#1} entered}]}}

%% ----------------------------------------------------------------------------
%%  Function Documentation

\newenvironment{functiondoc}[4][]{\dofndoc[#1]{#2}{#3}{#4}{black}}%
{\endfndoc}

\newenvironment{depfunctiondoc}[4][]{\dofndoc[#1]{Deprecated~#2}{#3}{#4}{red}}%
{\endfndoc}

\T\begin{ifhtml}
  \newcommand{\fnidxoff}{}
  \newcommand{\subsubentities}{}
  \newcommand{\nofndocindex}{\renewcommand{\fnidxoff}{t}}
  
  \newcommand{\dofndoc}[5][]{\renewcommand{\fntype}{#2}%
    \renewcommand{\fnname}{\textcolor{#5}{#3}}%
    \renewcommand{\fnarglist}{#4}%
    \EmptyP{#1}{\xname{ref-#1}}{\xname{ref-#3}}%
    \EmptyP{\subsubentities}{\subsubsection*[\textcolor{#5}{#3}]{}}%
                            {\subsection*[\textcolor{#5}{#3}]{}}%
    \label{ent:#3}%
    \EmptyP{\fnidxoff}{\newcommand{\fnidxoff}{}}{\bfindex{#3}}}
  
  \newcommand{\endfndoc}{}
  
  \newcommand{\fnsyntax}{\endpar\par
    \xml{table class="tight" width="100%"}
      \xml{tr valign="top"}%
      \xml{td align="left"%}
      \xml{nobr}\textbf{\fnname}\xml{/nobr}%
      \xml{/td}%
      \xml{td} \xmlent{nbsp} \xml{/td}%
      \xml{td align="left" width="99%"}%
        \fnarglist
        \xml{/td}%
        \xml{td align="right"}%
        [\textit{\fntype}]
        \xml{/td}%
        \xml{/tr}%
        \xml{/table}%
        \endpar}

 \newcommand{\alternatetarted}{}

 \newcommand{\fnalternate}[2]{%
     \EmptyP{\alternatestarted}{\xml{table class="tight" width="100%"}}{\xml{table class="tighttop" width="100%"}}
      \renewcommand{\alternatestarted}{*}%
      \xml{tr valign="top"}%
      \xml{td align="left"}%
      \xml{nobr}#1\xml{/nobr}%
      \xml{/td}%
      \xml{td}\xmlent{nbsp} \xml{/td}%
      \xml{td align="left" width="99%"}
        #2
      \xml{/td}%
      \xml{/tr}%
      \xml{/table}}
  \newcommand{\endfnalternate}{}

  \newcommand{\fnsetfsyntax}[3]{\xml{table class="tighttop" width="100%"}
      \xml{tr valign="top"}
      \xml{td align="left"}
        \xml{nobr}(setf (#1\xml{/nobr}
        \xml{/td}
        \xml{td}\xmlent{nbsp} \xml{/td}
        \xml{td align="left" width="99%"}
          #2) #3)
          \xml{/td}
          \xml{/tr}
          \xml{/table}}
  \newcommand{\endfnsetfsyntax}{}

\T\end{ifhtml}

\W\begin{iftex}
\makeatletter

\newcounter{fnpartctr}

\newcommand{\ckfnpartctr}[2]{\ifnum\the\value{fnpartctr}>#1\errmessage{#2 is
  out of sequence}\fi
  \setcounter{fnpartctr}{#1}}

\def\fndocrule{\textcolor{rulegray}{\rule[2pt]{\textwidth}{2pt}}}
\newcounter{fnstartpage}
\newif\iffnidx\fnidxtrue
\newcommand{\nofndocindex}{\fnidxfalse}

\newcommand{\dofndoc}[5][]{\thispagestyle{fancybottom}%
  \global\def\fntype{#2}%
  \global\def\fnname{#3}%
  \global\def\fnarglist{#4}%
  \global\def\fnnamecolor{#5}%
  \setcounter{fnpartctr}{0}%
  \setcounter{fnstartpage}{\c@page}%
  \if \fnnamecolor black
    \addcontentsline{toc}{subsubsection}{\textbf{#3}}%
  \else
    \addcontentsline{toc}{subsubsection}{\textcolor{\fnnamecolor}{\textbf{#3}}}%
  \fi
  \markright{#3}%
  \iffnidx\bfindexstart{#3}\fi}

\def\fnsyntax{\fndocrule\\[2pt]
  \settowidth{\tlengtha}{\textbf{\fnname}}%      % length of function-name
  \settowidth{\tlengthb}{[\textit{\fntype\/}]}%  % length of [type] part
  \setlength{\tlengthc}{\hsize}%
  \addtolength{\tlengthc}{-\tlengtha}%           % subtract function-name length
  \addtolength{\tlengthc}{-\tlengthb}%           % subtract [function] part
  \addtolength{\tlengthc}{-10pt}%                % subtract a bit more whitespace
  \if \fnnamecolor black
    \mbox{\textbf{\fnname}}~{\parbox[t]{\tlengthc}%
      {\nohyphenation\raggedright\textrm{\frenchspacing\fnarglist\hfil}}}%
  \else
    \mbox{\textcolor{\fnnamecolor}{\textbf{\fnname}}}~{\parbox[t]{\tlengthc}%
      {\nohyphenation\raggedright\textrm{\frenchspacing\fnarglist\hfil}}}%
  \fi
  \hfill[\textit{\fntype\/}]\\[2pt]\fndocrule\\
  \label{ent:\fnname}}%

\def\fnalternate#1#2{%
  \settowidth{\tlengtha}{#1}%                    % length of function-name
  \setlength{\tlengthc}{\hsize}%
  \addtolength{\tlengthc}{-\tlengtha}%           % subtract function-name length  
  \vskip -3pt
  \par
  \mbox{#1}~{\parbox[t]{\tlengthc}%
    {\nohyphenation\raggedright\textrm{\frenchspacing #2\hfil}}}}%

\def\fnsetfsyntax#1#2#3{%
  \settowidth{\tlengtha}{(setf (#1}%             % length of function-name
  \setlength{\tlengthc}{\hsize}%
  \addtolength{\tlengthc}{-\tlengtha}%           % subtract function-name length  
  \par
  \mbox{(setf (#1}~{\parbox[t]{\tlengthc}%
    {\nohyphenation\raggedright\textrm{\frenchspacing #2) #3)\hfil}}}}%

\def\endfndoc{\nopagebreak\fndocrule\\\nopagebreak
  \ifnum\value{fnstartpage}=\c@page\else
  \if@twoside 
    \ifodd\c@page
      \hspace*{1in}\hfill\textcolor{darkergray}{\textbf{\fnname}}%
    \else
    \textcolor{darkergray}{\textbf{\fnname}}%
  \fi\fi\fi
  \iffnidx\bfindexend{\fnname}\else\global\fnidxtrue\fi
  \clearpage}

\makeatother
\W\end{iftex}

%% ----------------------------------------------------------------------------

\W\begin{iftex}
\makeatletter

\def\functionsyntax#1#2{%
  \leftindentby[\exampleindent]%
  \settowidth{\tlengtha}{\textbf{#1}}%           % length of function-name
  \setlength{\tlengthc}{\linewidth}%
  \addtolength{\tlengthc}{-\tlengtha}%           % subtract function-name length
  \addtolength{\tlengthc}{-3pt}%                 % subtract a bit more whitespace
  \vskip -\baselineskip
  \mbox{\textbf{#1}~{\parbox[t]{\tlengthc}%
      {\baselineskip 0.85\baselineskip
        \nohyphenation\raggedright\textrm{\frenchspacing #2\hfil}}}}%
  \hfill\par\endleftindent}

\makeatother
\W\end{iftex}

%% ----------------------------------------------------------------------------

\newcommand{\returns}{$\Rightarrow${}}
\newcommand{\expands}{$\Longrightarrow${}}
\newcommand{\nil}{\code{nil}}


\W\begin{iftex}
%% This is the desired order in each functiondoc entry
\newcommand{\fnpurpose}{\ckfnpartctr{1}{Purpose}%
 \fndocsec{Purpose}~\\[2pt]\nopagebreak}
\newcommand{\fnsetf}{\ckfnpartctr{2}{Setf}%
  \fndocsec{Setf syntax}~\\[2pt]\nopagebreak}
\newcommand{\fnmethods}{\ckfnpartctr{3}{Method signatures}%
  \fndocsec{Method signatures}~\\[4pt]\nopagebreak}
\newcommand{\fnpackage}{\ckfnpartctr{4}{Package}%
  \fndocsec{Package}}
\newcommand{\fnmodule}{\ckfnpartctr{5}{Module}%
  \fndocsec{Module}}
\newcommand{\fnvaluetype}{\ckfnpartctr{5}{Value type}%
  \fndocsec{Value type}}
\newcommand{\fninitialvalue}{\ckfnpartctr{6}{Initial value}%
  \fndocsec{Initial value}}
\newcommand{\fnargs}{\ckfnpartctr{7}{Arguments}%
  \fndocsec{Arguments}~\\[-2pt]\nopagebreak}
\newcommand{\fnreturns}{\ckfnpartctr{8}{Returns}%
  \fndocsec{Returns}~\\[2pt]\nopagebreak}
\newcommand{\fnevents}{\ckfnpartctr{9}{Events}%
  \fndocsec{Events}~\\[2pt]\nopagebreak}
\newcommand{\fnerrors}{\ckfnpartctr{10}{Errors}%
  \fndocsec{Errors}~\\[2pt]\nopagebreak}
\newcommand{\fndsyntax}{\ckfnpartctr{11}{Detailed syntax}%
  \fndocsec{Detailed syntax}~\\[2pt]\nopagebreak}
\newcommand{\fndsyntaxwgray}{\ckfnpartctr{11}{Detailed syntax}%
  \fndocsec{Detailed syntax}~\\[2pt]\nopagebreak
  \hfil\textsf{\footnotesize\textcolor{darkergray}{[Syntax shown in gray is
        not supported in GBBopen Version~\gbbopenversion, but will become
        available in a future release.]}}\\[6pt]}
\newcommand{\fnterms}{\ckfnpartctr{12}{Terms}%
  \fndocsec{Terms}~\\[2pt]\nopagebreak}
\newcommand{\fndescription}{\ckfnpartctr{13}{Description}%
  \fndocsec{Description}~\\[2pt]\nopagebreak}
\newcommand{\fnalsos}{\ckfnpartctr{14}{See also}%
  \fndocsec{See also}~\\[2pt]\nopagebreak}
\newcommand{\fnexample}{\ckfnpartctr{14}{Example}%
  \fndocsec{Example}~\\[2pt]\nopagebreak} 
\newcommand{\fnexamples}{\ckfnpartctr{15}{Examples}%
  \fndocsec{Examples}~\\[2pt]\nopagebreak}
\newcommand{\fnnote}{\ckfnpartctr{16}{Notes}%
  \fndocsec{Note}~\\[2pt]\nopagebreak}
\newcommand{\fnnotes}{\ckfnpartctr{16}{Notes}%
  \fndocsec{Notes}~\\[2pt]\nopagebreak}

\let\syntaxsep=\par

\newenvironment{args}[1]{%
  \settowidth{\tlengtha}{{\var{#1\/}}}%
  \tlengthb=\tlengtha
  \advance\tlengthb by 0.5em
  \list{}{\topsep=5pt
    \partopsep=5pt
    \labelsep=0.5em
    \itemsep=-2pt
    \parskip=0pt
    \labelwidth=\tlengtha
    \leftmargin=\tlengthb
    \let\arg=\item
    \let\makelabel=\var}}{\endlist}

\newenvironment{keywords}[1]{%
  \settowidth{\tlengtha}{{\code{#1}}}%
  \tlengthb=\tlengtha
  \advance\tlengthb by 0.5em
  \list{}{\topsep=5pt
    \partopsep=5pt
    \labelsep=0.5em
    \itemsep=-2pt
    \parskip=0pt
    \labelwidth=\tlengtha
    \leftmargin=\tlengthb
    \let\arg=\item
    \let\makelabel=\code}}{\endlist}

\newcommand{\keyword}[1][]{\item[#1]}%

\newenvironment{alsos}[1]{\fnalsos
  \settowidth{\tlengtha}{{\textbf{#1}}}%
  \tlengthb=\tlengtha
  \advance\tlengthb by 0.5em
  \advance\tlengthb by \parindent
  \list{}{\topsep=5pt
    \partopsep=5pt
    \labelsep=0.5em
    \itemsep=-2pt
    \parskip=0pt
    \labelwidth=\tlengtha
    \leftmargin=\tlengthb
    \let\makelabel=\textbf}}{\endlist}

\newcommand{\also}[1][]{\item[#1] (page~\pageref{ent:#1})}%

\newcommand{\secalso}[2][]{\item[\rm #1\hfil] (page~\pageref{#2})}%

\makeatletter
\newcommand\fndspar{\@startsection{paragraph}{4}{\z@}%
  {-1ex \@plus -.2ex \@minus -.1ex}%
  {1pt}%
  {\normalfont\normalsize\bfseries}}
\makeatother
\W\end{iftex}

\T\begin{ifhtml}
\newcommand{\separ}{\par\endpar}

\newcommand{\fnpurpose}{\par
  \xml{span class="fndoclabel"}Purpose\xml{/span} \\}
\newcommand{\fnsetf}{\separ
  \xml{span class="fndoclabel"}Setf syntax\xml{/span}} 
\newcommand{\fnmethods}{\separ
  \renewcommand{\alternatestarted}{}%
  \xml{span class="fndoclabel"}Method signatures\xml{/span}}
\newcommand{\fnpackage}{\separ
  \xml{span class="fndoclabel"}Package\xml{/span}~~~}
\newcommand{\fnmodule}{\separ
  \xml{span class="fndoclabel"}Module\xml{/span}~~~}
\newcommand{\fnvaluetype}{\separ
  \xml{span class="fndoclabel"}Value type\xml{/span}~~~} 
\newcommand{\fninitialvalue}{\separ
  \xml{span class="fndoclabel"}Initial value\xml{/span}~~~} 
\newcommand{\fnargs}{\separ
  \xml{span class="fndoclabel"}Arguments\xml{/span}} 
\newcommand{\fnreturns}{\par
  \xml{span class="fndoclabel"}Returns\xml{/span} \\} 
\newcommand{\fnevents}{\par
  \xml{span class="fndoclabel"}Events\xml{/span} \\} 
\newcommand{\fnerrors}{\par
  \xml{span class="fndoclabel"}Errors\xml{/span} \\} 
\newcommand{\fndsyntax}{\par
  \xml{span class="fndoclabel"}Detailed syntax\xml{/span}}  
\newcommand{\fndsyntaxwgray}{\par
  \xml{span class="fndoclabel"}Detailed syntax\xml{/span}\\
  \xml{img height=14 width=0 src="shim.gif"}
  \textsf{\small\textcolor{darkergray}{[Syntax shown in gray is
        not supported in GBBopen Version~\gbbopenversion, but will become
        available in a future release.]}}}
\newcommand{\fnterms}{\separ
  \xml{span class="fndoclabel"}Terms\xml{/span}}  
\newcommand{\fndescription}{\par
  \xml{span class="fndoclabel"}Description\xml{/span} \\} 
% \fndetails is only for HyperLaTeX descriptions
\newcommand{\fndetails}{\par
  \xml{span class="fndoclabel"}Details\xml{/span} \\} 
\newcommand{\fnalsos}{\separ
  \xml{span class="fndoclabel"}See also\xml{/span}}
\newcommand{\fnexample}{\separ
  \xml{span class="fndoclabel"}Example\xml{/span} \\} 
\newcommand{\fnexamples}{\separ
  \xml{span class="fndoclabel"}Examples\xml{/span} \\} 
\newcommand{\fnnote}{\par
  \xml{span class="fndoclabel"}Note\xml{/span} \\} 
\newcommand{\fnnotes}{\par
  \xml{span class="fndoclabel"}Notes\xml{/span} \\} 
\newcommand{\argstarted}{}
\newcommand{\syntaxsep}{\\}
\newenvironment{args}[1]{%
  \xml{table class="tighttop"}%
  \renewcommand{\argstarted}{}}%
{\xml{/td}\xml{/tr}\xml{/table}}%
\newcommand{\arg}[1][]{\EmptyP{\argstarted}{\xml{/td}\xml{/tr}}{}%
  \renewcommand{\argstarted}{*}%
  \xml{tr valign="top"}%
  \xml{td}\xml{i}#1\xml{/i}\xml{/td}%
  \xml{td}~~~~\xml{/td}%
  \xml{td}}
\newcommand{\keystarted}{}
\newenvironment{keywords}[1]{%
  \xml{table class="tight"}%
  \renewcommand{\keystarted}{}}%
{\xml{/td}\xml{/tr}\xml{/table}}%
\newcommand{\keyword}[1][]{\EmptyP{\keystarted}{\xml{/td}\xml{/tr}}{}%
  \renewcommand{\keystarted}{*}%
  \xml{tr valign="top"}%
  \xml{td}\xml{code}#1\xml{/code}\xml{/td}%
  \xml{td}~\xml{/td}%
  \xml{td}}
\newenvironment{alsos}[1]{\fnalsos}{}%
\newcommand{\also}[1][]{\xml{br}~~~~\link{\textbf{#1}}{ent:#1}}%
\newcommand{\secalso}[2][]{\xml{br}~~~~\link{#1}{#2}}%
\newcommand{\fndspar}[1]{\xml{p}\textbf{#1}\xml{/p}}
\T\end{ifhtml}

%% ----------------------------------------------------------------------------
%%  Glossary

\W\begin{iftex}
\newenvironment{glossary-list}{
  \list{}{\topsep=0pt
    \labelsep=0.5em
    \itemsep=0pt
    \parskip=0pt
    \let\makelabel=\textbf}}{\endlist}
\newcommand{\glent}[1][]{\item[#1]~\\}
\newcommand{\glref}[1]{#1}
\newcommand{\gllabel}[1]{}
\W\end{iftex}

\T\begin{ifhtml}
\newenvironment{glossary-list}{\begin{description}}{\end{description}}
\newcommand{\glent}[1][]{\xml{p}\item [\label{gl:#1}\textbf{#1}]}
\newcommand{\glref}[1]{\link{#1}{gl:#1}}
\newcommand{\gllabel}[1]{\label{gl:#1}}
\T\end{ifhtml}

%% ----------------------------------------------------------------------------
%%  TightItemize

\W\begin{iftex}
\makeatletter
\def\tightitemize{%
  \ifnum \@itemdepth >\thr@@\@toodeep\else
    \advance\@itemdepth\@ne
    \edef\@itemitem{labelitem\romannumeral\the\@itemdepth}%
    \expandafter
    \list
      \csname\@itemitem\endcsname
      {\def\makelabel##1{\hss\llap{##1}}%
        \itemsep -1pt
        \topsep\z@
        \partopsep\z@
        }%
  \fi}
\def\endtightitemize{\endlist}
\makeatother
\W\end{iftex}

\T\begin{ifhtml}
\newenvironment{tightitemize}{\endpar\HlxBlk\xml{ul class="tight"}\begingroup
  \newcommand{\item}{\HlxBlk\xml{li}}\ignorespaces}{\endgroup
  \HlxBlk\xml{/ul}}
\T\end{ifhtml}

%% ----------------------------------------------------------------------------
%%  TightEnumerate

\W\begin{iftex}
\makeatletter
\def\tightenumerate{%
  \ifnum \@enumdepth >\thr@@\@toodeep\else
    \advance\@enumdepth\@ne
    \edef\@enumctr{enum\romannumeral\the\@enumdepth}%
      \expandafter
      \list
        \csname label\@enumctr\endcsname
        {\usecounter\@enumctr\def\makelabel##1{\hss\llap{##1}}%
        \itemsep -1pt
        \topsep\z@
        \partopsep\z@
        }%
  \fi}
\def\endtightenumerate{\endlist}
\makeatother
\W\end{iftex}

\T\begin{ifhtml}
\newenvironment{tightenumerate}{\endpar\HlxBlk\xml{ol class="tight"}\begingroup
  \newcommand{\item}{\HlxBlk\xml{li}}\ignorespaces}{\endgroup
  \HlxBlk\xml{/ol}}
\T\end{ifhtml}

%% ----------------------------------------------------------------------------
%%  NoHyphenation

\W\begin{iftex}
\makeatletter

\def\nohyphenation{\pretolerance=10000\tolerance=10000}
\def\nohyphenationpar{\par\nohyphenation}
\def\endnohyphenationpar{\par}

%% ----------------------------------------------------------------------------
%%  AvoidHyphenation

\def\avoidhyphenation{\@ifnextchar[{\@avoidhyphen}{\@avoidhyphen[4]}}
\def\@avoidhyphen[#1]{\pretolerance=550\tolerance=550
   \multiply \pretolerance #1
   \multiply \tolerance #1}
\def\avoidhyphenationpar{\@ifnextchar[{\par\@avoidhyphen}%
   {\par\@avoidhyphen[4]}}
\def\avoidnohyphenationpar{\par}

\makeatother
\W\end{iftex}

%% ----------------------------------------------------------------------------
%%  \hhmm Time of Day (from Nelson Beebe)

\W\begin{iftex}
\newcount\hh
\newcount\mm
\mm=\time
\hh=\time
\divide\hh by 60
\divide\mm by 60
\multiply\mm by 60
\mm=-\mm
\advance\mm by \time
\def\hhmm{\number\hh:\ifnum\mm<10{}0\fi\number\mm}
\W\end{iftex}

%% ----------------------------------------------------------------------------
%%  Hyperlatex Emacs-Debugging Help

%\T\begin{ifhtml}
%\HlxEval{
%(defun hyperlatex-format-buffer-0 ()
%  (progn 
%    (hyperlatex-format-buffer-1)
%    (hyperlatex-warning-summary)
%    0))
%}
%\T\end{ifhtml}

%% ----------------------------------------------------------------------------
%% Remove Hyperlatex Emacs version & date-and-time-string comment from
%% each generated page (to reduce repository changes)

\T\begin{ifhtml}
\HlxEval{

(defun hyperlatex-make-node-header (head)
  "Creates header for new node, with filename, title etc."
  (delete-region (point-min) (point))
  (setq hyperlatex-current-filename
	(concat hyperlatex-html-directory "/"
		(hyperlatex-fullname hyperlatex-node-number)))
  (setq hyperlatex-made-panel hyperlatex-make-panel)
;; RK's change (replace UTF-8 with hyperlatex-xml-charset)
;;  (hyperlatex-gen "?xml version=\"1.0\" encoding=\"UTF-8\"?" "\n")
  (hyperlatex-gen 
    (concat "?xml version=\"1.0\" encoding=\"" hyperlatex-xml-charset "\"?") 
    "\n")
  ;; XML intro is user defined
  (if hyperlatex-xml
      ()
    (hyperlatex-gen
     (concat
      "!DOCTYPE html PUBLIC \"-//W3C//DTD XHTML 1.0 Transitional//EN\"\n"
      "   \"DTD/xhtml1-transitional.dtd\"")
     "\n")
    (hyperlatex-gen "html xmlns=\"http://www.w3.org/1999/xhtml\"" "\n"))
  (hyperlatex-gen
   (concat "!-- XML file produced from " hyperlatex-produced-from
	   "\n     using Hyperlatex v "
	   hyperlatex-version " (c) Otfried Cheong"
           ;; Remove Emacs version & current time string -- DDC
	   ; "\n     on Emacs " emacs-version ", " (current-time-string) 
           " --")
   "\n")
  (if hyperlatex-xml
      (let ((start (point)))
	(insert "\\HlxXmlIntro{}\n")
	(hyperlatex-format-region start (point)))
    (hyperlatex-gen (hyperlatex-get-attributes "head") "\n")
    (hyperlatex-gen (hyperlatex-get-attributes "title"))
    (insert hyperlatex-title 
	    (hyperlatex-purify (if head (concat " -- " head) "")))
    (hyperlatex-gen "/title" "\n")
    (if hyperlatex-final-pass
	(let ((start (point)))
	  (insert "\\HlxStyleSheet{}\n\\HlxMetaFields{}\n")
          (let ((nav-string "\274img"))
            ;; add base to the Navigation (TOC) section:
            (if (equal (and (>= (length head) 4)
                            (substring head 0 4))
                            nav-string)
               ;; IE requires base to be in <HEAD>
               (hyperlatex-gen "base target=\"right\"\n")))
	  (hyperlatex-format-region start (point))))
    (hyperlatex-gen "/head" "\n")
    (hyperlatex-gen (hyperlatex-get-attributes "body") "\n"))
  (setq hyperlatex-label-number 1)
  (setq hyperlatex-node-section hyperlatex-sect-number)
  (if (and hyperlatex-final-pass hyperlatex-made-panel)
      (let ((start (point)))
	(insert "\\HlxTopPanel{}\n")
	(hyperlatex-format-region start (point))))
  (hyperlatex-format-resumepars))

(defun hyperlatex-make-frames-headers ()
  "Creates frameset file."
  ;; first we make the frameset
  (delete-region (point-min) (point))
;; RK's change
  (hyperlatex-gen "?xml version=\"1.0\" encoding=\"UTF-8\"?" "\n")
;;  (hyperlatex-gen 
;;    (concat "?xml version=\"1.0\" encoding=\"" hyperlatex-xml-charset "\"?") 
;;    "\n")
  (hyperlatex-gen
   (concat "!DOCTYPE html PUBLIC \"-//W3C//DTD XHTML 1.0 Frameset//EN\"\n"
           "   \"DTD/xhtml1-frameset.dtd\"")
   "\n")
  (hyperlatex-gen "html xmlns=\"http://www.w3.org/1999/xhtml\"" "\n")
  (hyperlatex-gen
   (concat "!-- XML file produced from " hyperlatex-produced-from
	   "\n     using Hyperlatex v "
	   hyperlatex-version " (c) Otfried Cheong"
           ;; Remove Emacs version & current time string -- DDC
	   ;"\n     on Emacs " emacs-version ", " (current-time-string) 
           " --")
   "\n")
  (hyperlatex-gen (hyperlatex-get-attributes "head") "\n")
  (hyperlatex-gen (hyperlatex-get-attributes "title"))
  (insert hyperlatex-title)
  (hyperlatex-gen "/title" "\n")
  (hyperlatex-gen "/head")
  (let ((begin (point)))
    (insert "\\HlxFramesDescription{"
	    hyperlatex-basename "}{" hyperlatex-html-ext "}")
    (hyperlatex-gen "/html" "\n")
    (let ((end (point-marker)))
      (hyperlatex-format-region begin end)
      (goto-char end)
      (set-marker end nil)))
;; RK's
  (hyperlatex-debug-show-p-positions)
;;
  ;; save the node
  (save-restriction
    (narrow-to-region (point-min) (point))
    (hyperlatex-final-substitutions)
    (hyperlatex-write-region (point-min) (point-max)
			     (concat hyperlatex-html-directory "/"
				     hyperlatex-basename hyperlatex-html-ext))
    (goto-char (point-max))))
}
\T\end{ifhtml}

%% ----------------------------------------------------------------------------
%%  Hyperlatex \hhmm

\T\begin{ifhtml}
\HlxEval{
(put 'HlxHHMM 'hyperlatex 'hyperlatex-HHMM)

(defun hyperlatex-HHMM ()
   (let* ((date (decode-time))
          (minutes (elt date 1))
          (hours (elt date 2)))
      (insert
         (concat (int-to-string hours)
             ":" (format "%02d" minutes)))))
}
\newcommand{\hhmm}{\HlxHHMM}
\T\end{ifhtml}

%% ----------------------------------------------------------------------------
%%  Hyperlatex \par control hacks

\T\begin{ifhtml}
\HlxEval{
(put 'supp 'hyperlatex 'Hlxsupp)
(put 'notpretop 'hyperlatex 'Hlxnotpretop)

(setq ddc-supp nil)
(setq notpretop nil)

(defun Hlxsupp ()
  (setq ddc-supp t))

(defun Hlxnotpretop ()
  (setq notpretop t))

(defun hyperlatex-leave-par ()
;; Leave the current paragraph (if in any) and go into vertical mode.
  (if ddc-supp
      (progn (setq ddc-supp nil)
             ;(hyperlatex-gen "br" "\n")
             )
      (if (not hyperlatex-in-b) nil
        (if (hyperlatex-in-par-p) (hyperlatex-gen "/p" "\n"))
        (if (hyperlatex-get-state) ; if state is defined the stack is non-empty
          (hyperlatex-mode-set-state (cons hyperlatex-v-mode (point)))))))

(defun hyperlatex-format-example ()
  (hyperlatex-format-endpar)
;; RK's
  (hyperlatex-mode-level-up "example")
;;
  (if hyperlatex-in-body 
      (progn
	(hyperlatex-format-suspendpars)
	(setq hyperlatex-verbatim-need-pars t))
    (setq hyperlatex-verbatim-need-pars nil))
  (hyperlatex-blk)
  (hyperlatex-gen (if (not notpretop)
                      "pre class=\"pretop\""
                      (progn (setq notpretop nil)
                             "pre")))
  (let ((hyperlatex-special-chars-regexp
	 (concat "[\\\\{}%" hyperlatex-meta-| "]"))
	(hyperlatex-example-depth hyperlatex-recursion-depth)
	(hyperlatex-active-space t))
    ;; recursive call returns after processing \end{example}
    (hyperlatex-format-region (point) (point-max) t))
  (goto-char hyperlatex-command-start)
;  (hyperlatex-delete-whitespace)
  (hyperlatex-gen "/pre")
;; RK's
  (hyperlatex-mode-level-down "example")
;;
  (if hyperlatex-verbatim-need-pars
      (hyperlatex-format-resumepars)))

(defun hyperlatex-format-verbatim ()
  (hyperlatex-format-endpar)
;; RK's
  (hyperlatex-mode-level-up "format-verbatim")
;; end-of-RK's
  (if hyperlatex-in-body 
      (progn
	(hyperlatex-format-suspendpars)
	(setq hyperlatex-verbatim-need-pars t))
    (setq hyperlatex-verbatim-need-pars nil))
  (hyperlatex-delete-|)
  (let ((env-name (symbol-name (car hyperlatex-stack))))
    (hyperlatex-blk)
    (hyperlatex-gen (if (not notpretop)
                        "pre class=\"pretop\""
                        (progn (setq notpretop nil)
                               "pre")))
    (search-forward  (concat "\\end{" env-name "}"))
    (goto-char (match-beginning 0))))

}
\newcommand{\supp}{\Hlxsupp}
\newcommand{\notpretop}{\Hlxnotpretop}
\T\end{ifhtml}

%% ----------------------------------------------------------------------------
%%  Hyperlatex tabular class="tight" Mod

\T\begin{ifhtml}
\HlxEval{
(put 'tabletop 'hyperlatex 'Hlxtabletop)

(setq tabletop nil)

(defun Hlxtabletop ()
   ;; DDC mod to switch table class (hack):
   (setq tabletop 't))

(defun hyperlatex-format-tabular ()
  (hyperlatex-parse-optional-argument)
  (setq hyperlatex-tabular-column-descr
	(cons (cons 0 (hyperlatex-tabular-posn
		       (hyperlatex-parse-required-argument)))
	      hyperlatex-tabular-column-descr))
  (hyperlatex-format-endpar)
  (hyperlatex-blk)
  ;; Add "tight" class DDC
  ; (hyperlatex-gen (hyperlatex-get-attributes "table"))
  (hyperlatex-gen (concat (hyperlatex-get-attributes "table")
                          (if tabletop " class=\"tighttop\""
                                       " class=\"tight\"")))
  (setq tabletop nil)
  ;; End "tight" class mod DDC
  (hyperlatex-gen (hyperlatex-get-attributes "tbody"))
  (hyperlatex-gen "tr" "\n")
  ;; Do not put "\n" after <td> because it disturbs paragraph recognition.
  ;; (A "\n" after <td> may precede one from the input, after \begin{tabular}, 
  ;; which thus would mean a paragraph ending command.)
  ;; Introducing a meta-character for "\n" would help for such cases.
  (hyperlatex-gen
   (format (concat "%s colspan=" hyperlatex-meta-dq "1" hyperlatex-meta-dq
		   " align=" hyperlatex-meta-dq "%s" hyperlatex-meta-dq)
	   (hyperlatex-get-attributes "td")
	   (hyperlatex-tabular-cell-align)))
;; RK's
  (hyperlatex-mode-level-up "format-tabular")
;; end of RK's
  ;; put cell text into a group to restore attributes like \bf at its end
  (hyperlatex-begin-group)
  ;; put contents of >{...} in front of the cell
  (hyperlatex-tabular-cell-front))
}
\newcommand{\tabletop}{\Hlxtabletop}
\T\end{ifhtml}

%% ----------------------------------------------------------------------------
%%  Hyperlatex remove Navigation section from Navigation menu and force
%%  inclusion of Tools and Core subsections:

\T\begin{ifhtml}
\HlxEval{
(defun special-section-p (secname)
  "DDC added predicate for GBBopen Reference `special' subsections" 
  (member secname
	  '(;; :gbbopen-tools
            "Declared Numerics"
            "Date and Time"
            "Offset Universal Time"
	    "Portable Threads"
	    "Polling Functions"
	    "Portable Sockets"
	    "OS Interface"
            ;; :gbbopen-core
            "Links"
            "Events"
            "Intervals"
            "Blackboard Repository"
            "Instance Retrieval"
            "Saving and Sending"
	    "Queue Management")))

(defun hyperlatex-insert-menu (secnum depth)
  "Insert a menu for section SECNUM of depth DEPTH."
  (let ((sp hyperlatex-rev-sections))
    (hyperlatex-blk)
    ;; search sp for a secction numbered secnum
    (while (/= (hyperlatex-sect-num (car sp)) secnum)
      (setq sp (cdr sp)))
    ;; sp points to section 
    (let* ((vis nil) ;; visited levels of sp
           (lev (hyperlatex-sect-level (car sp)))
           ;; DDC added:
	   (node-secname (third (car sp))))
      (setq sp (cdr sp))
      (while (and sp (> (hyperlatex-sect-level (car sp)) lev))
        ;; sp points to a subsection of mine!
        
        ;; DDC mod ---
;	(if (> (hyperlatex-sect-level (car sp)) (+ lev depth))
	(let* ((secname (third (car sp)))
               (special-section-p (special-section-p secname))
               ;; The Navigation section starts with "<img":
	       (nav-string "\274img")
               (sect-level (hyperlatex-sect-level (car sp))))
          (if (or (and (> sect-level (+ lev depth))
                       (not special-section-p))
                  ;; don't add the Navigation section itself to the 
                  ;; navigation menu:
                  (equal (and (>= (length secname) 4)
                              (substring secname 0 4))
                         nav-string)
                  ;; don't add the special section entries to the entries
                  ;; listed on the Tools & GBBopen Core pages:
                  (and (or (string= node-secname "Tools")
                           (string= node-secname "GBBopen Core"))
                       special-section-p))
              ;; --- end DDC mod
              
              (setq sp (cdr sp)) ;; skip as too deep                     
              ;; make a menu entry
              (if (null vis)
                  (progn
                    (hyperlatex-gen (hyperlatex-get-attributes "ul") "\n")
                    (hyperlatex-gen (hyperlatex-get-attributes "li"))
                    (hyperlatex-insert-menu-item (car sp))
                    (setq vis (cons (hyperlatex-sect-level (car sp)) nil))
                    (setq sp (cdr sp)))
                  (if (= (car vis) (hyperlatex-sect-level (car sp)))
                      (progn
                        (hyperlatex-gen "/li" "\n")
                        (hyperlatex-gen (hyperlatex-get-attributes "li"))
                        (hyperlatex-insert-menu-item (car sp))
                        (setq sp (cdr sp)))
                      (if (< (car vis) (hyperlatex-sect-level (car sp)))
                          (progn
                            (insert "\n")
                            (hyperlatex-gen (hyperlatex-get-attributes "ul") "\n")
                            (hyperlatex-gen (hyperlatex-get-attributes "li"))
                            (hyperlatex-insert-menu-item (car sp))
                            (setq vis (cons (hyperlatex-sect-level (car sp)) vis))
                            (setq sp (cdr sp)))
                          ;; else: (> (car vis) (hyperlatex-sect-level (car sp)))
                          (progn
                            (hyperlatex-gen "/li" "\n")
                            (hyperlatex-gen "/ul" "\n")
                            (setq vis (cdr vis))))))))
        
        ;; DDC mods ---
        ;; match parens from above mod:
        )
      
      (while vis
        (hyperlatex-gen "/li" "\n")
        (hyperlatex-gen "/ul" "\n")
        (setq vis (cdr vis))))))
}
\T\end{ifhtml}

%% ----------------------------------------------------------------------------
%%  Hyperlatex - Christopher League's dash and dquote mods

\T\begin{ifhtml}
\HlxEval{
  (defvar hyperlatex-special-chars-basic-regexp
     (concat "\\\\%{}]\\|---?\\|``?\\|''?\\|\\?`\\|!`\\|"
	      hyperlatex-meta-|))
 
  (put '-  'hyperlatex-active 'hyperlatex-active-dash)

  (defun hyperlatex-active-tilde ()
     (hyperlatex-gensym "nbsp"))
 
  (defun hyperlatex-active-dash ()
     (let ((prechar (char-before (1- (point)))))
        (cond ((= prechar ?-)
	       (delete-char -2)
               (hyperlatex-gensym "mdash"))
              (t
               (delete-char -1)
               (hyperlatex-gensym "ndash")))))

  (defun hyperlatex-active-quote ()
     (let ((prechar (preceding-char)))
       (cond ((= prechar ?')
              (delete-char -1)
              (hyperlatex-gensym "#8221"))
             (t
              (hyperlatex-gensym "#8217")))))
 
  (defun hyperlatex-active-backquote ()
     (let ((prechar (preceding-char)))
       (cond ((= prechar ?`)
              (delete-char -1)
              (hyperlatex-gensym "#8220"))
             ((= prechar ??)
              (delete-char -1)
 	      (hyperlatex-gensym "#191"))
             ((= prechar ?!)
              (delete-char -1)
              (hyperlatex-gensym "#161"))
             (t
              (hyperlatex-gensym "#8216"))))) 
}
\T\end{ifhtml}

%% ----------------------------------------------------------------------------

%%  Hyperlatex makeidx.hlx extensions for subentry support 
%%  Extensions Version 1.03
%%  December 3 2004
%%  Dan Corkill <corkill@cs.umass.edu>
%%
%%  The following changes to makeidx.hlx (Version 1.0) support hyperlatex
%%  indexing with subentries (via "!" char) and commands (via "|" char).  As
%%  with normal LaTeX indexing, the double-quote character can be used to
%%  quote special characters.
%%
%%  To use, load this file after makeidx.hlx.
%%
%%  Note that since hyperlatex entries do not have page numbers
%%  (unless hyperlatex-index-node-format is true--the new default),
%%  entries involving a command do not generate an automatic \xlink
%%  (so the ! command must handle the xlink arguments itself). 
%%  For example, to do italic page numbers, \index{alpha|itidx},
%%  the command definitions for itidx would be:
%%     \W\newcommand{\itidx}[2]{\textit{\xlink{#1}{#2}}}
%%     \T\newcommand{\itidx}[1]{\textit{#1}}
%%  A command like \index{beta|see{alpha}} (where \see is already
%%  defined for LaTeX indexing) would call \see with 3 arguments in
%%  hyperlatex: (1) alpha, (2) beta, and (3) the url. (This definition
%%  of \see appears below.)
%%
%%  Some examples:
%%     \index{aaa"!bbb!ccc}%
%%     \index{aaa!bbb"!ccc}%
%%     \index{aaa!bbb!ccc@\textbf{ccc}}%
%%     \index{aaa!bbb@\textbf{bbb}!ccc}%
%%     \index{aaa@\textbf{aaa}!bbb!ccc}%
%%     \cindex[aaa!bbb!ccc]{aaa!bbb!\textbf{ccc}}%
%%     \cindex[aa!bb!cc]{aa!\textbf{bb}!\textit{cc}}%
%%     \cindex[aa"!bb!cc]{aa"!\textbf{bb}!\textit{cc}}%
%%     \index{aaa|see{aaa"!bbb}}%
%%     \index{aaa|itidx}%
%%     \index{aaa"!bbb"|ccc""ddd|see{wow}}%
%%     \index{aaa"!bbb"|ccc""ddd@fff|see{wow}}%
%%     \index{aaa"!bbb"|ccc""ddd"@fff|see{wow}}%

\T\begin{ifhtml}
%% \see definition for nil hyperlatex-index-node-format:
%% \W\newcommand{\see}[3]{#2, \textit{see} #1}
\W\newcommand{\see}[3]{\textit{see} #1}

\HlxEval{
(defvar hyperlatex-index-node-format t)

(defun hyperlatex-se-format-index ()
  "Adds an index entry."
  (let ((opt (hyperlatex-single-line (hyperlatex-parse-optional-argument)))
        (arg (hyperlatex-single-line (hyperlatex-parse-required-argument)))
        (label (hyperlatex-drop-label)))
    (unless hyperlatex-final-pass
      (unless opt 
        ;; If opt is specified, no @ commands should be present
        (let ((entry arg)
              (subentry nil)
              (subsubentry nil)
              entry-opt
              subentry-opt
              subsubentry-opt
              (pos (subentry-match arg))) 
          ;; break up entry into sub & subentries
          (cond 
           (pos (setq entry (substring arg 0 pos))
                (setq arg (substring arg (+ 1 pos)))
                (setq subentry arg)
                (setq pos (subentry-match arg))
                (cond
                 (pos (setq subentry (substring arg 0 pos))
                      (setq subsubentry (substring arg (+ 1 pos)))
                      ;; check subsubentry for @
                      (setq pos (atentry-match subsubentry))
                      (cond
                       (pos (setq subsubentry-opt 
                              (substring subsubentry 0 pos))
                            (setq subsubentry
                              (substring subsubentry (+ 1 pos))))
                       (t (setq subsubentry-opt subsubentry)))))
                ;; check subentry for @
                (setq pos (atentry-match subentry))
                (cond
                 (pos (setq subentry-opt 
                        (substring subentry 0 pos))
                      (setq subentry
                        (substring subentry (+ 1 pos))))
                 (t (setq subentry-opt subentry)))))
          ;; check entry for @
          (setq pos (atentry-match entry))
          (cond
           (pos (setq entry-opt 
                  (substring entry 0 pos))
                (setq entry
                  (substring entry (+ 1 pos))))
           (t (setq entry-opt entry)))
          (cond 
           (subsubentry
            (setq arg (concat entry "!" subentry "!" subsubentry))
            (setq opt (concat entry-opt "!" subentry-opt "!" subsubentry-opt)))
           (subentry
            (setq arg (concat entry "!" subentry))
            (setq opt (concat entry-opt "!" subentry-opt)))
           (t
            (setq arg entry)
            (setq opt entry-opt)))))
      (setq hyperlatex-index
        (cons (list (if opt opt arg) arg hyperlatex-node-number label)
              hyperlatex-index)))))

(defun hyperlatex-se-format-printindex ()
  (if (not hyperlatex-final-pass)
      ()
    (setq hyperlatex-index
          (sort (reverse hyperlatex-index)  ; Make sort more stable
               'hyperlatex-se-index-compare))
    (let ((indexelts hyperlatex-index)
          (used-chars (make-vector 256 nil))
          (toc-num    0)
          (symbol-num 1)
          (number-num 2)
          (char-num   3)
          (prev-num  -1)
          (first-in-section? t)
          (prev-entry "")
          (entry1 nil)
          (entry2 nil)
          this-num
          this-char)
      ;;
      ;; insert index body
      ;;
      (insert "\\begin{description}\n")
      (while indexelts
        (setq this-char (aref (upcase (car (car indexelts))) 0))
        ;;
        ;; determine entry type (numeric, alphabetic, symbol)
        ;;
        (cond ((and (<= ?0 this-char) (<= this-char ?9)
                    (string-match "^[0-9]+$" (car (car indexelts))))
               (setq this-num number-num))
              ((and (<= ?A this-char) (<= this-char ?Z))
               (setq this-num (+ (- this-char ?A) char-num)))
              (t
               (setq this-num symbol-num)))
        ;;
        ;; insert entry
        ;;
        (unless (= prev-num this-num)
          (aset used-chars this-num t)
          (let ((keyname this-char))
            (cond ((= this-num symbol-num)
                   (setq keyname "\\HlxIdxSymbols"))
                  ((= this-num number-num)
                   (setq keyname "\\HlxIdxNumbers")
                   (if (= prev-num symbol-num)
                       (insert "\\xml{p}"))))
            (insert "\\item[")
            (hyperlatex-se-insert-url toc-num this-num keyname)
            (insert "]\n")
            (setq first-in-section? t)
            (setq entry1 nil entry2 nil)
            (setq prev-num this-num)))
        (let* ((el (nth 1 (car indexelts)))
               (pos (subentry-match el))
               entry)
          (cond (pos
                 (setq entry (substring el 0 pos))
                 (unless (string= entry entry1)
                   (subentry-indent nil nil first-in-section?)
                   (setq first-in-section? nil)
                   (insert (format "%s\n" 
                                   (strip-quotes-from-string
                                    (setq entry1 entry)))))
                 (setq el (substring el (+ 1 pos)))
                 ;; sub-sub-entry?
                 (setq pos (subentry-match el))
                 (cond (pos
                        (setq entry (substring el 0 pos))
                        (unless (string= entry entry2)
                          (subentry-indent entry1 nil first-in-section?)
                          (insert (format "%s\n" 
                                          (strip-quotes-from-string
                                           (setq entry2 entry)))))
                        (setq el (substring el (+ 1 pos))))
                       (t (setq entry2 nil))))
                (t (setq entry1 nil)
                   (setq entry2 nil)))
          (let* ((pos (commandentry-match el))
                 (node (nth 2 (car indexelts)))
                 (url (hyperlatex-gen-url node
                                          (nth 3 (car indexelts)))))
            (cond (pos 
                   (cond (hyperlatex-index-node-format
                          (let ((ent 
                                 (strip-quotes-from-string  
                                  (substring el 0 pos))))
                            (unless (string= ent prev-entry)
                              (subentry-indent entry1 entry2 first-in-section?)
                              (insert ent)
                              (setq prev-entry ent)))
                          (insert ", ")
                          (insert (format "\\%s{%s}{%s}\n" 
                                          (strip-quotes-from-string
                                           (substring el (+ 1 pos)))
                                            node
                                            url)))
                         (t (subentry-indent entry1 entry2 first-in-section?)
                            (insert (format "\\%s{%s}{%s}\n" 
                                            (strip-quotes-from-string
                                             (substring el (+ 1 pos)))
                                            (strip-quotes-from-string 
                                             (substring el 0 pos))
                                            url)))))
                  (t (cond 
                      (hyperlatex-index-node-format
                       (let ((ent (strip-quotes-from-string el)))
                         (unless (string= ent prev-entry)
                           (subentry-indent entry1 entry2 first-in-section?)
                           (insert ent)
                           (setq prev-entry ent)))
                       (insert ", \n")
                       (insert (format "\\xlink{%s}{%s}" node url)))
                      (t (subentry-indent entry1 entry2 first-in-section?)
                         (insert (format "\\xlink{%s}{%s}"
                                         (strip-quotes-from-string el)
                                         url))))))))
        (setq first-in-section? nil)
        (setq indexelts (cdr indexelts)))
      (insert "\\end{description}\n")
      ;;
      ;; insert index directory
      ;;
      (goto-char hyperlatex-command-start)
      (insert "\\HlxBeginIndexDir{}")
      (let ((sep "")
            (name toc-num))
        (if (aref used-chars symbol-num)
            (progn
              (insert sep "\\HlxIndexDirActive{")
              (hyperlatex-se-insert-url symbol-num name "\\HlxIdxSymbols")
              (insert "}")
              (setq sep "\\HlxIndexDirSep{}")
              (setq name nil)))
        (if (aref used-chars number-num)
            (progn
              (insert sep "\\HlxIndexDirActive{")
              (hyperlatex-se-insert-url number-num name "\\HlxIdxNumbers")
              (insert "}")
              (setq sep "\\HlxIndexDirSep{}")
              (setq name nil)))
        (setq this-char ?A)
        (setq this-num char-num)
        (while (<= this-char ?Z)
          (if (aref used-chars this-num)
              (progn
                (insert sep "\\HlxIndexDirActive{")
                (hyperlatex-se-insert-url this-num name
                                          (char-to-string this-char))
                (insert "}")
                (setq name nil))
            (insert sep "\\HlxIndexDirInactive{" this-char "}"))
          (setq sep "\\HlxIndexDirSep{}")
          (setq this-char (1+ this-char))
          (setq this-num (1+ this-num))))
      (insert "\\HlxEndIndexDir{}\n")
      (goto-char hyperlatex-command-start)))
  ;; allocate label numbers
  (setq hyperlatex-label-number (+ hyperlatex-label-number 29)))

(defun subentry-indent (entry1 entry2 first-in-section?)
  (unless first-in-section? (insert "\\\\"))
  (cond (entry2 (insert "~~~~~~~~~~~~"))
        (entry1 (insert "~~~~~~"))))

(defun atentry-match (str &optional start)
  (let ((pos (string-match "@" str (or start 0))))
    (cond ((and pos (char-equal ?\" (aref str (- pos 1))))
           (atentry-match str (+ pos 1)))
          (t pos))))

(defun subentry-match (str &optional start)
  (let ((pos (string-match "!" str (or start 0))))
    (cond ((and pos (char-equal ?\" (aref str (- pos 1))))
           (subentry-match str (+ pos 1)))
          (t pos))))

(defun commandentry-match (str &optional start)
  (let ((pos (string-match "|" str (or start 0))))
    (cond ((and pos (char-equal ?\" (aref str (- pos 1))))
           (commandentry-match str (+ pos 1)))
          (t pos))))

(defun strip-quotes-from-string (str)
  (let* ((ptr 0)
         (end (length str))
         (offset 0))
    (while (< ptr end)
      (let ((char (aref str (+ ptr offset))))
        (when (and (char-equal ?\" char) 
                   ; ignore an erroneous quote at end of string
                   (< ptr (decf end)))
          (incf offset)
          (setq char (aref str (+ ptr offset))))
        (aset str ptr char)
        (incf ptr)))
    (substring str 0 end)))

;;; From makeindex(1):
;;;
;;; Modified to compare hyperlatex-index-node numbers on identitcal
;;; matches
;;;
;;; Numbers are always sorted in numeric order. Letters are first
;;; sorted without regard to case; when words are identical, the
;;; uppercase version precedes its lowercase counterpart.
;;;  
;;; A special symbol is defined here to be any character not appearing
;;; in the union of digits and the English alphabetic characters.
;;; Patterns starting with special symbols precede numbers, which
;;; precede patterns starting with letters. As a special case, a string
;;; starting with a digit but mixed with non-digits is considered to be
;;; a pattern starting with a special character.

(defun hyperlatex-se-index-compare (a b)
  (let* ((au (upcase (car a)))
         (bu (upcase (car b)))
         (a0 (aref au 0))
         (b0 (aref bu 0)))
    (cond
     ;; a is a number
     ((and (<= ?0 a0) (<= a0 ?9) (string-match "^[0-9]+$" au))
      (cond
       ((and (<= ?0 b0) (<= b0 ?9)      ; b is a number
             (string-match "^[0-9]+$" bu))
        (or (< (string-to-number au)
               (string-to-number bu))
            (and (= (string-to-number au)
                    (string-to-number bu))
                 ;; differentiate on node number
                 (< (nth 2 a) (nth 2 b)))))
       ((and (<= ?A b0) (<= b0 ?Z)) t)  ; b0 is alphabetic
       (t nil)))                        ; b0 is a symbol
     ;; a0 is alphabetic
     ((and (<= ?A a0) (<= a0 ?Z))
      (cond
       ((and (<= ?A b0) (<= b0 ?Z))     ; b0 is alphabetic
        (or (string< au bu)
            (and (string= au bu)
                 (or (string< (car a) (car b))
                     (and (string= (car a) (car b))
                          ;; differentiate on node number
                          (< (nth 2 a) (nth 2 b)))))))
       (t nil)))                        ; b0 is a digit or symbol
     ;; a0 is a symbol
     (t
      (cond
       ((and (<= ?0 b0) (<= b0 ?9)) t)  ; b0 is a digit
       ((and (<= ?A b0) (<= b0 ?Z)) t)  ; b0 is alphabetic
       (t (or (string< (car a) (car b)) ; b0 is a symbol
              (and (string= (car a) (car b))
                   ; differentiate  on node number
                   (< (nth 2 a) (nth 2 b))))))))))

}
\T\end{ifhtml}
  
%%% Local Variables: 
%%% mode: latex
%%% End: 
